\documentclass{article}

\usepackage{titlesec}
\usepackage{subcaption}
\usepackage{longtable}
\usepackage{booktabs}
\usepackage{amsmath}
\usepackage[normalem]{ulem} 
\usepackage[T1,T2A]{fontenc}
\usepackage[utf8]{inputenc}
\usepackage[english,russian]{babel}

\author{SDL}

\titleformat{\section}
            {\normalfont\Large}
            {}
            {0pt}
            {Урок \thesection\quad}

\begin{document}
\noindent\makebox[\textwidth]{\rule{\paperwidth}{0.4pt}}
\section{(07.09.2018)}
\noindent\makebox[\textwidth]{\rule{\paperwidth}{0.4pt}}

\subsection{Сборники тестов по русскому}
\begin{itemize}
\item
  Цыбулько - Подготовка к ЕГЭ по русскому языку
\item
  Егораева - Подготовка к ЕГЭ, Сборник заданий
\item
  Сенина - Подготовка к ЕГЭ
\end{itemize}

\paragraph{Орфографические задания} включают в себя задания с 9 по 15.

\subsection{Задание №9. Правописание гласных в корне}

\begin{enumerate}
\item
  Проверяемая гласная в корне. (ронять - урон, цветение - цвет)
\item
  Непроверяемая гласная в корне. (бетон, фантазии)
\item
  Чередующаяся гласная в корне.
\end{enumerate}

\paragraph{Чередующаяся гласная в корне}
\begin{enumerate}
\item
  \textbf{Чет-чит, бер-бир, мер-мир, пер-пир, стел-стил, блест-блист, жег-жиг, дер-дир, тер-тир, ня-ни, ча-чи.}
  Если после корня \^{а}, пишем ``и''. (замереть - замирать, отпереть - отпирать, замять - заминать).
  Исключение: сочетать - сочетание.
\item
  \textbf{Лож - лаг, кос - кас.} Если есть \^{а}, пишем ``а''. (положить - полагать, коснуться - касаться).
  Исключение: п\'{о}лог.
\item
  \textbf{Клан - клон, твар - твор, гар - гор.} О - в безударных корнях. (тварь - творчество - сотворить, кланяться -
  поклониться, загар - загореть). Исключения: \'{у}тварь, выгарки, пригарь.
\item
  \textbf{Зор - зар.} ``а'' в безударном положении. (зори - зарево - озарение). Исключение: зорянка, заревать.
\item
  \textbf{Плав - плов - плыв.} ``о''/''ы'' только в исключениях. (поплавок - плавучий).
  Исключения: пловец, пловчиха, плывун.
\item
  \textbf{Рост - раст - ращ.} Перед ``ст'', ``щ'' - а, в остальных случаях - о. (возраст - выращенный, водоросль).
  Исключения: Ростов, росток, Ростислав, ростовщик, отрасль
\item
  \textbf{Скач - скак - скоч.} ``Скач'' и ``скак'' обозначают многократное действие, ``скоч'' обозначает однократное..
  (проскакать - выскочить). Исключения: скачок, скачи
\item
  \textbf{Мак - мок.} ``Мак'' имеет значение ``опускать в жидкость'',
  ``мок'' - ``пропускать жидкость''. (обмакнуть кисть, непромокаемый плащ)
\item
  \textbf{Равн - ровн.} ``Равн'' в словах со значением ``одинаковый'', ``сходный''. ``Ровн'' в значении ``гладкий'',
  ``прямой''. (уравнение, подровнять). Исключения: уровень, ровестник, ровня, поровну, равнина.
\end{enumerate}

\noindent\makebox[\linewidth]{\rule{\paperwidth}{0.4pt}}
\section{(14.09.18)}
\noindent\makebox[\linewidth]{\rule{\paperwidth}{0.4pt}}
\paragraph{}

Не надо путать омонимичные корни:

\begin{enumerate}
\item
  \textbf{мер} - из\textbf{мер}яемый
\item
  \textbf{мир} - с\textbf{мир}ение
\item
  \textbf{кос} - по\textbf{кос}ился
\item
  \textbf{гор} - \textbf{гор}евать
\end{enumerate}

\paragraph{Словарный диктант (проверяемые гласные)}

Уплотнить сроки, 
приласкать собаку, 
обнажить пороки, 
раскалить сковороду, 
усложнить обстановку, 
облокатиться о перила, 
отдалить поездку, 
поглощать энергию, 
отказаться от услуг, 
угрожать расправой,  
оправдать поступок, 
вопиющий произвол, 
юный запивала, 
зарядить ружье,
роптать на судьбу,
смягчение приговора,
увядание красоты,
истинное заглядение,
высокое заграждение,
зачарованный взгляд,
обрамление картины,
нагревание предмета,
лечебное голодание,
неподдельное удивление,
чистосердечное покаяние,
вечное смирение,
внезапное озлобление,
невыносимое угнетение,
далёкая сторона,
упрощенный пример

\begin{itemize}
\item
  \textbf{Полногласие}: оло, оро, еле, ере. (молоко, корова, берег, ворота)
\item
  \textbf{Неполногласие}: ра, ле, ре. (млеко, брег, врата)
\end{itemize}

\paragraph{}

Если дано слово с полногласием, его надо проверять словом с полногласием.
\paragraph{}
Если слово с неполногласием, его надо проверять словом с неполногласием

\paragraph{Словарный диктант (чередующаяся гласная)}
Блистательный, блестящий, обжигание, пробираться, расстелить, расстилаться, сочетать, касаться, неукоснительно,
предложить, безотлагательный, зарница, озарять, погорелец, сотворение, отклонение, плавучий, пловец, вскочить,
обскакать, нарастание, произрастать, отскочить, взрослеть, наклонять, Ростислав, излагать, обложение, замереть,
отпирать, растворить, утварь

\paragraph{Найти слово с проверяемой безударной гласной в корне}

Замереть, поплавок, ветеран, росток, \underline{примирить врагов}




\paragraph{}
\noindent\makebox[\linewidth]{\rule{\paperwidth}{0.4pt}}
\section{(21.09.18)}
\noindent\makebox[\linewidth]{\rule{\paperwidth}{0.4pt}}

\paragraph{Задание 11.} Правописание приставок. Правописание и/ы после приставок.
  Разделительный мягкий и твердый знаки

\paragraph{Приставки на з/c - без, воз, вз, из, низ, раз, чрез, через}
Если после приставки следует глухой согласный - пишем с.\\
Пример: ра\underline{с}свет, и\underline{с}чезать, ни\underline{с}провергать, чере\underline{с}чур\\

Если после приставки следует звонкий согласный - пишем з.\\
Пример: ра\underline{з}бить, и\underline{з}дать, ни\underline{з}вергать, чре\underline{з}вычайный\\

! приставки з не существует (\underline{с}дать, \underline{с}шить)\\

В словах здесь, здоровье, здание, зга буква з является частью корня.\\

\paragraph{Приставки раз(с) - роз(с)}
Рассказ - россказни,
расписать - роспись, 
разлить - розлив, 
развалить - розвальни

\paragraph{Неизменяемые приставки над, об, от, под, пред, в}
Отдел, отчёт, обман, обсыпать

\paragraph{Приставка пра со значением "предок, прошлое"}
Пр\underline{а}бабушка, пр\underline{а}язык. НО пр\underline{о}образ

\paragraph{Правописание букв ы/и после приставок}
ы пишется после русских твёрдых приставок (предыстория, разыскать, отыграть)\\
и пишется после русских приставок меж и сверх (межинститутский, сверхизысканный)\\
и пишется после иностранных приставок пан, суб, пост, дез, контр, транс (постимпрессионизм)\\
и пишется в сложносокращенных словах (самиздат, пединститут)\\
Запомнить! взимать, но изымать

\paragraph{Словарный диктант (с/з)}

Восхождение на вершину, расчётный счёт, чересчур много, восполнить пропущенное,
безграничное доверие, пламенное воззвание, расформировать полк, истратить деньги,
чрезмерный восторг, искоренить недостатки, иссякший источник, бесперспективный план,
ниспровергать авторитеты, разжечь костёр, безветренный вечер, бесформенная масса,
бесжалостный поступок, бессмертное произведение, бессрочный отпуск, воспитать ученика,
воспроизвести текст, возрастной ценз, восседать на престоле, резкий тон, сдвинутые шкафы,
сдержать слово, низшая порода, распаковать вещи.

\paragraph{Словарный диктант (ы/и)}

Безыскусственный, небезызвестный, предыстория, сверхизысканный, безысходный,
межинститутский, взимать, изымать, субинспектор, трансиорданский, возыметь,
обыскать, дезинформация, обындеветь, подытожить, постимпрессионизм, сверхиндустриализация,
спортигра, предыюльский, сызмала, отымённый, панисламизм, фининспектор, межигровой.

\newpage
\noindent\makebox[\linewidth]{\rule{\paperwidth}{0.4pt}}
\section{(28.09.18)}
\noindent\makebox[\linewidth]{\rule{\paperwidth}{0.4pt}}

\subsection{Приставки пре- и при-}
\paragraph{}

\textbf{Пре-}
\begin{enumerate}
\item
  Когда обозначает "очень, высокая степень чего-либо". Примеры: пр\underline{е}возносить,
  пр\underline{е}увеличивать, пр\underline{е}мудрый.
\item
  Когда можно заменить приставкой пере-. Примеры: пр\underline{е}дание, пр\underline{е}ступление,
  пр\underline{е}вратник, пр\underline{е}емник.
\end{enumerate}
\paragraph{}

\textbf{При-}
\begin{enumerate}
\item
  Значение близости, присоединения, начала. Примеры: пр\underline{и}станционный, пр\underline{и}ступить,
  пр\underline{и}писать, Пр\underline{и}морье.
\item
  Неполнота действия (можно подставить наречие "чуть-чуть"). Примеры: пр\underline{и}открыть,
  пр\underline{и}спустить, пр\underline{и}утихнуть.
\item
  Доведение действия до предела. Примеры: пр\underline{и}искать, пр\underline{и}выкнуть,
  пр\underline{и}учить, пр\underline{и}ручить.
\item
  Действие в собственных интересах. Примеры: пр\underline{и}манить, пр\underline{и}нарядиться,
  пр\underline{и}слушаться (к совету), пр\underline{и}своить.
\item
  Сопутствующее действие. Примеры: пр\underline{и}прыгивать, пр\underline{и}свистывать,
  пр\underline{и}танцовывать.
\end{enumerate}
\paragraph{}

\paragraph{Русские словарные слова}

\paragraph{Пре-} Пр\underline{е}словутый, пр\underline{е}имущество, непр\underline{е}минуть,
пр\underline{е}пинание, пр\underline{е}рекаться, пр\underline{е}кословить, пр\underline{е}ткновение,
пр\underline{е}тить.

\paragraph{При-} Без пр\underline{и}крас, непр\underline{и}ступный, пр\underline{и}скорбный,
пр\underline{и}вередлиный, пр\underline{и}гожий, пр\underline{и}личный, пр\underline{и}язнь,
пр\underline{и}тязание, воспр\underline{и}ятие, пр\underline{и}звание, пр\underline{и}сутствовать,
воспр\underline{и}емник.

\paragraph{Иностранные словарные слова}

\paragraph{Пре-} Пр\underline{е}амбула, пр\underline{е}валировать, пр\underline{е}зидент,
пр\underline{е}зумпция, пр\underline{е}мьера.

\paragraph{При-} Пр\underline{и}ватный, пр\underline{и}вилегии, пр\underline{и}митивный,
пр\underline{и}оритет.

\begin{longtable}[c]{|p{3cm}|p{3cm}|}
  \caption{Парные слова по контексту}\\
  \hline
  пр\underline{е}дать друга & пр\underline{и}дать форму шара\\
  \hline
  предел мечтаний & придел в церкви\\
  \hline
  Презрение к трусу & призрение сирот\\
  \hline
  преклоняться перед красотой & приклонить ветки\\
  \hline
  претворить планы в жизнь & притвориться больным\\
  \hline
  претерпеть изменения & притерпеться к запаху\\
  \hline
  преходящий (временный) & приходящая няня\\
  \hline
  непреходящий (вечный) & приходящая няня\\
  \hline
  пребывать в городе & прибывать в город\\
  \hline
\end{longtable}

\subsection{Разделительные ъ/ь}

\paragraph{}
\textbf{Ъ}

\begin{enumerate}
\item
  После твёрдых \textbf{приставок}. Примеры: об\underline{ъ}езд, кон\underline{ъ}юнктура,
  ад\underline{ъ}ютант.
\item
  После корней \textbf{двух}, \textbf{трёх}, \textbf{четырёх}. Примеры: двух\underline{ъ}ядерный.
\end{enumerate}

\paragraph{}
\textbf{Ь}
\begin{enumerate}
\item
  В корне слова. Примеры: в\underline{ь}юга, обез\underline{ь}яны, б\underline{ь}ют.
\item
  В притяжательных прилагательных перед и, е. Примеры: птич\underline{ь}и, помещич\underline{ь}ему.
\item
  В существительных множественного числа перед и, я, е. Примеры: сынов\underline{ь}я, руч\underline{ь}и.
\item
  В иноязычных словах на о. Примеры: буль\underline{о}н, почталь\underline{о}н, медаль\underline{о}н,
  ожерел\underline{ь}е.
\end{enumerate}

\paragraph{Запомнить слово: фельд\underline{ъ}егерь.}

\paragraph{Диктант на пре- при-} Прибывать в город, пребывать в неволе, преемник традиций, радиоприёмник,
предание старины глубокой, придать блеск поверхности, презирать негодяя, призреть бездомных детей, претворить
планы в жизнь, притворить дверь, камень преткновения, приткнуться в уголок, преходящий момент, приходящая няня,
претерпеть лишения, притерпеться к боли, предел мечтаний, придел в храме, преподать урок, припадать к плечу,
приклонить ветки, преклонить колени, пресветлый образ, прискорбный файт, преемлимый вариант,
приемственность поколений.

\noindent\makebox[\linewidth]{\rule{\paperwidth}{0.4pt}}
\section{(05.10.18)}
\noindent\makebox[\linewidth]{\rule{\paperwidth}{0.4pt}}

\paragraph{Диктант на разделительные Ъ и Ь} Партьер, пьеса, бульон, предъявить, изъян, обезьяна, съязвить,
лисья, разъяренный, эскадрилья, трёхэтажный, сверхъестественный, мы бьём, въявь, сыновья, сэкономить,
съездить, мелколесье, подъём, объявление, сагитировать, арьергард, объективный, вьюга, межъярусный,
пьедестал, предъюбилейный, бильярд, отъявленный, фортепиано, безъядерный, рьяный, интерьер, предъявитель.

\paragraph{}

Первые четыре задания: даётся микротекст.
\begin{enumerate}
\item
  Выбрать два предложения, называющие главную мысль текста.
\item
  Вставить слово в пропуск.
\item
  Угадать слово по контексту.
\end{enumerate}

\paragraph{Образец заданий 1-3.}
\begin{enumerate}
\item
Чаще всего преломление лучей света в воздухе незначительно, а вот ложка в стакане чая кажется нам сломанной.
\item
  Причина заключается в различной плотности воды и воздуха.
\item
  ... переходя из одной среды в другую, лучи света преломляются, изменяя прямолинейный путь, отклоняясь по закону
физики в сторону более плотной среды.
\end{enumerate}

Вставить в пропуск: наоборот, \underline{поэтому}, так как, зато, вследствие.\\

\paragraph{}

\subsection{Четвертое задание --- расстановка ударения. (см. индекс ударений)}

\paragraph{Дополнение:}
\begin{enumerate}
\item
  Глаголы прошедшего времени женского рода: несл\'{а}, везл\'{а}.
  \textbf{НО}: кл\'{а}ла, кр\'{а}ла, стл\'{а}ла, рж\'{а}ла.
\item
  Т\'{о}рты, п\'{о}рты, аэропо\'{р}ты.
\item
  Сложные слова с корнем \textit{лог}. Некрол\'{о}г, катал\'{о}г, диал\'{о}г, монол\'{о}г.
\item
  Сложные слова с корнем \textit{вод}. Мусоропров\'{о}д, газопров\'{о}д, водопров\'{о}д.
  \textbf{НО}: электропр\'{о}вод.
\end{enumerate}

\paragraph{} Примеры заданий: 

\begin{itemize}
\item
  Прож\'{и}вший, д\'{о}верху, позвал\'{а}, сл\'{и}вовый, одобрен\'{а}.
\item
  З\'{а}гнутый, дон\'{е}льзя, откл\'{ю}ченный, приб\'{ы}в, ч\'{е}люстей.
\item
  Облил\'{а}сь, аэроп\'{о}рты, нед\'{у}г, опт\'{о}вый.
\item
  Убыстр\'{и}ть, моза\'{и}чный, д\'{о}верху, с\'{о}гнутый, жалюз\'{и}.
\item
  Пр\'{и}нятый, св\'{ё}кла, облегч\'{и}т, к\'{у}хонный, кор\'{ы}сть.
\item
  Ср\'{е}дство, пон\'{я}вший, окруж\'{и}т, кр\'{а}ны, исч\'{е}рпав.
\item
  Бухг\'{а}лтеров, вероиспов\'{е}дование, повтор\'{е}нный, обостр\'{и}ть, сир\'{о}ты.
\item
  Катал\'{о}г, полож\'{и}л, рвал\'{а}, окл\'{е}ить, прин\'{у}дить.
\end{itemize}

\subsection{23 задание --- типы речи}

\begin{enumerate}
\item
  Описание (портрет, пейзаж, интерьер). Основные части речи --- прилагательное, наречие, причастие.
\item
  Повествование (раскадровка). Часть речи --- глагол.
\item
  Рассуждение (нельзя передать изображением). Абстрактные понятия, специальные конструкции, наличие вводных слов,
  риторические восклицания, обращения, вопросы, сложноподчиненные предложения с придаточными причины, следствия,
  уступки, цели. Слова потому что, так что, хотя, чтобы.
\end{enumerate}

\newpage
\noindent\makebox[\linewidth]{\rule{\paperwidth}{0.4pt}}
\section{(12.10.18)}
\noindent\makebox[\linewidth]{\rule{\paperwidth}{0.4pt}}

\subsection{Задание №6. Стилистическое.}
\paragraph{}
\textbf{Плеон\'{а}зм} (греч. "излишество") --- многословие или выражения, содержащие близкие по смыслу слова.

\paragraph{Примеры} Каждая минута \sout{времени}, в апреле \sout{месяце}, \sout{промышленная} индустрия,
отступить \sout{назад}, \sout{своя} автобиография, \sout{впервые} знакомиться,\\
\sout{неожиданный} сюрприз, сувенир \sout{на память},
\sout{местный} абориген, \sout{белый} альбинос, \sout{первая} премьера, \sout{свободная} вакансия,
\sout{агрессивный} экстремизм, \sout{главный} приоритет, \sout{первый} лидер, \sout{тоска} по ностальгии,
\sout{полный} аншлаг, \sout{передовой} авангард, сто рублей \sout{денег}, прейскурант \sout{цен},
час \sout{времени}, \sout{совместное} сотрудничество, поступательное движение \sout{вперёд}, \sout{лично} я.

\paragraph{}

\textbf{Тавтология --- это}
\begin{enumerate}
\item
Повторение сказанного другими словами, не вносящими ничего нового. Пример: Авторские слова --- это слова автора.
\item
Повторение в предложении однокоренных слов. Пример: Следует отметить следующие особенности.
\item
Неоправданная избыточность выражения. Пример: более лучшее положение, самые высочайшие вершины.
\end{enumerate}

\subsection{Первый тип: исключите слово.}
\paragraph{}
Имя каталической монахини Матери Терезы сегодня ассоциируется с безграничной любовью к человечеству и
с \sout{бескорыстным} альтруизмом.

\subsection{Второй тип: замените слово.}
\paragraph{}
Вчера состоялся \underline{консилиум}, на котором учителя должны были решить, исключить из школы
хулигана Сидорова или дать ему возможность исправиться. (Заменить на педсовет)

\newpage
\subsection{Задание №7. Лексические ошибки.}

\begin{longtable}[c]{|p{3cm}|p{4cm}|p{4cm}|}
  \caption{Лексические ошибки}\\
  \hline
  \textbf{Части речи} & \textbf{Тип ошибки} & \textbf{Правильный вариант}\\
  \hline
  Имя существительное & Неправильное употребление падежа и форм на \textit{ч}. Падежов, туфлей, нет апельсин,
  нет армянов, мн. ч. шофера & Падежей, туфель, апельсинов, армян, шоферы\\
  \hline
  Местоимение & Неправильная форма с предлогами. Идти к ему, думать об ей, ихний, скучать по вам & К нем, о ней, их,
  скучать по вас.\\
  \hline
  Прилагательное, наречие & Неправильное образование форм степеней сравнения. Более легче, менее быстрее, менее
  быстрейший & Более легко, менее быстро, самый быстрый.\\
  \hline
  Числительное & Неправильный падеж числительного. Семистами, к две трети, в двух тысяч первом год, двое подруг,
  трое сестёр, по обоим сторонам & Семьюстами, к двум третям, в две тысячи первом году, две подруги, три сестры,
  по обеим сторонам\\
  \hline
  Глагол & Неправильные формы. Попробовает, жгёт, ложить, бежите, едь, ехай, езжай & Попробует, жжёт, положить, бегите,
  поезжай\\
  \hline
  Деепричастие & Неправильное образование деепричастия. Сделая, делав & Сделав, делая.\\
  \hline
\end{longtable}

\newpage
\noindent\makebox[\linewidth]{\rule{\paperwidth}{0.4pt}}
\section{(26.10.18)}
\noindent\makebox[\linewidth]{\rule{\paperwidth}{0.4pt}}

\subsection{Задание 8. Грамматические ошибки}

\begin{enumerate}
\item
  Ошибки в построении предложений с однородными членами.
  \begin{itemize}
  \item
    Употребление глаголов с разным управлением: Мы \emph{интересуемся} и любим \emph{посещать} выставки
    филателистов.\\
    \textbf{Правильно:} Мы интересуемся марками и любим посещать выставки филателистов.
  \item
    Неправильное употребление двойных сопоставительных союзов: Не только, а... Не столько, но...\\
    \textbf{Правильно:}
    Не только, но и... Не столько, сколько... Если не, то... Так же, как и... Как, так и...
  \end{itemize}
\item
  Неправильное употребление существительного с предлогом.
  \begin{itemize}
  \item
    \begin{equation*}
      \begin{cases}
        \text{Согласно}\\
        \text{Вопреки}\\
        \text{Благодаря}\\
        \text{Например}\\
        \text{Наперекор}
      \end{cases}
      \text{с родительным падежом}
    \end{equation*}
    \textbf{Правильно:} использование дательного падежа.
  \item
    По прибытию, по окончанию, по приезду, по прилёту\\
    \textbf{Правильно:} по прибытии, по окончании, по приезде, по прилёте
  \end{itemize}
\item
  Нарушение связи между подлежащим и сказуемым.
  \begin{itemize}
  \item
    Сочи \emph{строились}. Правильно: Сочи \emph{строился}.
  \item
    Все, кто \emph{был} в комнате, вдруг \emph{услышал} громкий звук.
    Все, кто \emph{были} в комнате, вдруг \emph{услышали} громкий звук.\\
    \textbf{Правильно:} Все, кто \emph{был} в комнате, вдруг \emph{услышали} громкий звук
  \end{itemize}
\item
  Нарушение в построении предложения с причастным оборотом.
  \begin{itemize}
  \item
    Неправильный падеж причастия: Драматурт ставит ряд вопросов, \emph{волнующие} зрителей.\\
    \textbf{Правильно:} Драматург ставит ряд вопросов, \emph{волнующих} зрителей.
  \item
    Неправильная конструкция: \emph{Осуждавшие} стихи \emph{не только убийцу, но и знать}, разошлись по всей России.\\
    \textbf{Правильно:} Стихи, \emph{осуждавшие не только убийцу}, но и знать, разошлись по всей России.
  \end{itemize}
\item
  Нарушение в построении предложения с несогласованным приложением.
  \begin{itemize}
  \item
    Мы любим читать газету "\emph{Вечернюю Москву}".\\
    \textbf{Правильно:} Мы любим читать газету "\emph{Вечерняя Москва}".\\
    \textbf{Замечание:} Слова \emph{город}, \emph{река} допускают употребление приложений в косвенных падежах:
    В городе Казани, по реке Оби.
  \end{itemize}
\item
  Неправильное построение предложения с косвенной речью.
  \begin{itemize}
  \item
    Нельзя употреблять местоимения первого и второго лица: Базаров говорил, что "\emph{мой} дед землю пахал".\\
    \textbf{Правильно:} Базаров говорил, что его дед землю пахал.
  \end{itemize}
\item
  Нарушение видовременных форм глагола.
  \begin{itemize}
  \item
    Нельзя в одном предложении употреблять формы разных времён: Он \emph{задумался} о смысле жизни и
    \emph{хочет} изменить её.\\
    \textbf{Правильно:} Он \emph{задумался} о смысле жизни и \emph{хотел} изменить её.
  \end{itemize}
\item
  Нарушение построения предложения с деепричастным оборотом.
  \begin{itemize}
  \item
    Отсутствие деятеля: \emph{Увидев} красный сигнал светофора, машина \emph{остановилась}.\\
    \textbf{Правильно:} \emph{Увидев} красный сигнал светофора, \emph{водитель остановил} машину.\\
    \textbf{Замечание}
    \begin{itemize}
    \item
      Уходя, (вы) гасите свет.
    \item
      Уходя, он погасил свет.
    \item
      Уходя, можно гасить свет.
    \end{itemize}
  \end{itemize}
\end{enumerate}

\newpage
\noindent\makebox[\linewidth]{\rule{\paperwidth}{0.4pt}}
\section{(09.11.18)}
\noindent\makebox[\linewidth]{\rule{\paperwidth}{0.4pt}}

\subsection{Задание 11. Правописание суффиксов глаголов}

\begin{enumerate}
\item
  \textbf{Ова/ева/ыва/ива}
  \begin{itemize}
  \item
    Если в I лице ед. ч. нет \emph{ва}, окончания \emph{ую}, \emph{юю}.\\
    \textbf{Пример:} командую - командовать.
  \item
    Если в I лице ед. ч. есть \emph{ва}, окончания \emph{ываю}, \emph{иваю}.\\
    \textbf{Пример:} опаздываю - опаздывать.
  \end{itemize}
\item
  Ударный суффикс \textbf{в\'{а}}\\
  Перед суффиксом \emph{ва} сохраняется та же гласная, что и гласная в парном по виду.\\
  \textbf{Примеры:} залить - заливать, запеть - запевать.\\
  \textbf{НО:} затмить - затмевать, продлить - продлевать, обуревать, увещевать, сомневаться, недоумевать.
\item
  Глаголы с приставками \textbf{обез} (\textbf{обес})
  \begin{itemize}
  \item
    \textbf{и} пишется в переходных глаголах (управляющих винительным падежом).\\
    \textbf{Примеры:} обезводить водоём, обессмертить имя.
  \item
    \textbf{е} пишется в непереходных глаголах.\\
    \textbf{Примеры:} местность обезлесела, больной обескровел.
  \end{itemize}
\item
  Суффикс \textbf{л}\\
  Перед суффиксами прошедшего времени \emph{л} сохраняется суффикс инфинитива.\\
  \textbf{Примеры:} думать - думал, сеять - сеял, обидеть - обидел.
\end{enumerate}

\paragraph{Диктант}
Заведовать отделом, оправдывать друга, исповедовать христианство, завидовать другу,
испытывать печаль, призывать к примирению, задумываться над судьбой, попробовать салат,
край обезлюдел, обессилел от тяжкого труда, солдат обескровил от раны, кочевать из города в город,
просачивалась в щели, гарцевать на коне, требовать еды, выбрасывали на песок,
лелеял мечту, слегка вздрагивать, по-настоящему обрадовались, размахивать руками,
обмакивать в воду, веяло прохладой, ненавидел с детства, просеяла муку,
обиделся на меня, опротивела ложь, переглядывались между собой, докладывал перед аудиторией,
перечитывать повесть, намеревался приехать, обесточить район, затеял уборку,
проветривать помещение, рассматривали фотографии, затмевать разум, обезболивать ожог.

\newpage
\noindent\makebox[\linewidth]{\rule{\paperwidth}{0.4pt}}
\section{(07.12.18)}
\noindent\makebox[\linewidth]{\rule{\paperwidth}{0.4pt}}

\subsection{Суффиксы прилагательных}

\begin{enumerate}
\item
  \textbf{ев/ив}\\
  Безударный - \emph{ев}, ударный - \emph{ив}\\
  Примеры: игривый, нулевой, никелевый.\\
  НО: милостивый, юродивый.
\item
  \textbf{чив/лив}\\
  Суффиксов \emph{чев/лев} нет.\\
  Примеры: настойчивый, назойливый\\
  НО: гуттаперчевый

\item
  \textbf{онь/еньк}\\
  После \emph{г, к, х} пишется \emph{оньк}, в остальных случаях \emph{еньк}.\\
  Примеры: глубоконький, сухонький, синенький

\item
  \textbf{оват/еват}\\
  После твердых согласных пишется \emph{оват, еват} пишется после мягких, шипящих и \emph{ц}.\\
  Примеры: горьковатый, голубоватый, рыжеватый.

\item
  \textbf{инск/енск}\\
  \emph{Инск} пишется
  \begin{itemize}
  \item
    в прилагательных, образованных от существительных женского рода с окончаниями \emph{а/я} (сестринский)
  \item
    в географических названиях на \emph{и, ы, а, я}. (сочинский)
  \end{itemize}
  \emph{Енск} пишется в остальных случаях. (Гродно --- гродненский)

\item
  \textbf{к/ск}\\
  \emph{к} пишется
  \begin{itemize}
  \item
    в прилагательных с основой на \emph{к/ц/ч} и с чередованием \emph{к,ч/ц}. (немец --- немецкий)
  \item
    в качественных прилагательных с беглым гласным \emph{о} в корне. (скользок --- скользкий)
  \end{itemize}
  \emph{ск} пишется в отымённых прилагательных с основой на \emph{д,т,з,с}. (завод --- заводской, тунгус --- тунгусский)
  
\item
  \textbf{чат}\\
  Суффикса \emph{чет} не существует.\\
  Пишется в прилагательных с основой на \emph{зд, с, ст, ш, т, ц/т}. (борозда --- бороздчатый, брус --- брусчатый)

\item
  \textbf{ат}\\
  Пишется после \emph{щ} и \emph{с} череводанием \emph{ст/щ, ск/щ}. (хруст --- хрущатый, доска --- дощатый)

\item
  \textbf{Запомнить!}\\
  Пензенский, пресненский, коломенский\\
  Узбекский, нью-йоркский, таджикский, угличский, владивостокский, тюркский
\end{enumerate}


\paragraph{Диктант} алюминиевый, волевой, гречневый, гуттаперчевый, долевой, доходчивый, дрожжевой, заботливый,
завистливый, замшевый, запасливый, засушливый, затейливый, изворотливый, каракулевый, кварцевый, ключевой, марлевый,
милостивый, непоседливый, неуживчивый, никелевый, обидчивый, переменчивый, питьевой, пищевой, плюшевый, предприимчивый,
придирчивый, сбивчивый, сиреневый, соевый, угодливый, угреватый, уклончивый, услужливый, уступчивый, фланелевый, щавелевый

\newpage
\noindent\makebox[\linewidth]{\rule{\paperwidth}{0.4pt}}
\section{(25.01.19)}
\noindent\makebox[\linewidth]{\rule{\paperwidth}{0.4pt}}

\subsection{Правописание не}

\subsubsection{Слитно}

\begin{enumerate}
\item
  Без не не употребляется  
\item
  В отрицательных местоимениях и наречиях под ударением (некто, негде, некогда).
\item
  В полных одиночных причастиях
\item
  В составе недо (неполнота) - недоеденный обед
\item
  В кратких причастиях с им (нелюбим, неделим)
\item
  В производных предлогах (несмотря на)
\item
  совершенно, абсолютно, слишком, крайне, весьма ...
\end{enumerate}

\subsubsection{Раздельно}

\begin{enumerate}
\item
  Противопоставление с союзом а, но если слова не антонимы - слитно
\item
  С глаголами, деепричастиями, числительными, конкр. существительными (не хотел, не думая, не три, не книга)
\item
  В отрицательных местоимениях с предлогом. (не с кем)
\item
  В отглагольных прилаг., прич, если есть зависимое слово.
\item
  Если есть частицы и союзы далеко не, вовсе не, отнюдь не, ничуть не, еще не, пока не, чуть ли не
\item
  С краткими прилагательными, причастиями
\end{enumerate}

\paragraph{Диктант}
Нелепый поступок,
невысоко взлететь,
не высоко взлететь, а низко,
путь не близок, а далек,
дорога не ровная, но короткая,
очень не короткий зонт,
весьма нерешительный шаг,
ничем не оправданный поступок,
далеко не легкое дело,
несмолкаемый гул,
не веселый, а грустный взгляд,
неблагодарный слушатель,
вовсе не трудовые доходы,
отвечать небрежно,
полный невежда в музыке,
вести себя непринужденно,
река была неширока,
путь не короток, а длинен,
неряшливый вид,
бормотал что-то невнятное,
почуять недоброе,
незавидное положение,
ненавистный человек,
обошлось недешево,
домашние неурядицы,
душевные невзгоды,
сказал явную неправду,
невзлюбить с первого взгляда

\newpage
\noindent\makebox[\linewidth]{\rule{\paperwidth}{0.4pt}}
\section{(22.02.19)}
\noindent\makebox[\linewidth]{\rule{\paperwidth}{0.4pt}}

\subsection{Задание 14. Слитное/дефисное/раздельное написание частей речи}

Через дефис пишутся:
\begin{enumerate}
\item
  Предлоги: из-за, из-под, по-за, по-над.
\item
  Наречия:
  \begin{enumerate}
  \item
    С приставкой по + ски, ему, цки\\
    Примеры: по-моему, по-немецки, по-волчьи
  \item
    От числительных в(во) + их\\
    Примеры: во-первых, в-третьих
  \item
    С повторяющейся основой\\
    Примеры: еле-еле, чуть-чуть, крест-накрест, точь-в-точь\\
    Исключения: точка в точку, с боку на бок, шаг за шагом, с глазу на глаз, в конце концов
  \item
    -то, -либо, -нибудь\\
    Примеры: так-то, когда-нибудь, кое-где
  \end{enumerate}
\item
  Местоимения: см (2d)\\
  Примеры: какой-то, что-нибудь, кое-кто
\item
  Существительные и прилагательные
  \begin{itemize}
  \item
    Название частей света: юго-запад, зюйд-ост, северо-запад
  \item
    Оттенки цвета и вкуса: кисло-сладкий, сине-зеленый
  \item
    Физические термины: киловатт-час, гамма-излучение
  \item
    Названия растений: иван-чай, мать-и-мачеха, перекати-поле
  \item
    Должности, звания: унтер-офицер, вице-президент
  \item
    Двойные фамилии и географические названия: Семёнова-Звенигородская, Орехово-Борисово
  \item
    Существительные с приложением: поэт-романтик, царь-пушка, диван-кровать
  \item
    Название партий или их членов: социал-демократ, национал-социалистический
  \item
    Термины: механико-математический, весенне-полевой, общественно-полезный
  \end{itemize}
\end{enumerate}
%TODO: Правописание слов с корнем пол

\newpage
\noindent\makebox[\linewidth]{\rule{\paperwidth}{0.4pt}}
\section{(01.03.2019)}
\noindent\makebox[\linewidth]{\rule{\paperwidth}{0.4pt}}

\subsection{Слова с корнем пол}
\begin{itemize}
\item
  Через дефис:
  \begin{enumerate}
  \item
    Второй корень начинается с гласной (пол-яблока, пол-ананаса)
  \item
    Второй корень начинается с буквы л (пол-лимона, пол-литра)
    Исключение: поллитровка
  \item
    Второй корень - имя собственное (пол-Парижа, пол-Москвы)
  \end{enumerate}
\item
  Слитно:
  \begin{enumerate}
  \item
    Второй корень начинается с согласной (полбатона)
  \end{enumerate}
\end{itemize}

\subsection{Слитное написание частей речи}

\begin{enumerate}
\item
  Предлоги: вследствие, ввиду, сверх, насчет, наподобие, вместо, вроде.
\item
  Наречие:
  \begin{itemize}
  \item
    Без приставки не употребляются (исподтишка, дотла)
  \item
    Приставка + наречие (никогда, подешевле, надолго)
  \item
    Образовано от кратких прилагательных (издавна, влево)
    Окончание определяется по правилу окна (в, на, за, из, до, с)
  \item
    От полных прилагательных и существительных с приставкой в (вдаль, вкрутую)
    Но: на босовую, на мировую, на попутную
  \item
    От числительных с приставками в, на (вдвое, натрое).
    Но: по двое, по трое.
  \end{itemize}
\item
  Существительные и прилагательные
  \begin{itemize}
  \item
    С соединительной гласное (землекоп, живопись)
  \item
    Глагол в повелительном наклонении (горехвостка, держиморда)
  \item
    Первый корень - числительное (пятилетний)
  \item
    Первый корень - существительное в именительном падеже (семядоля, летоисчисление, времяпрепровождение)
  \item
    Первый корень - иноязычный (мото, фото, био, нео)
  \item
    Прилагательные, образованные из словосочетания (железнодорожный, дальневосточный)
  \item
    Славянские корни, обозначающие параметры: высота, ширина, глубина (благородный, широкомасштабный, левобережный)
  \end{itemize}
\item
  Союзы
  \begin{enumerate}
  \item
    Тоже, также в значении и (Он тоже любит яблоки)
    Но: то же, что и; то же самое, одно и то же
  \end{enumerate}
\end{enumerate}

\subsection{Раздельное написание}

\begin{itemize}
\item
  Предлоги: в течение, в заключение, по мере, в связи, в продолжение, по причине, в силу, по прибытии, по истечении,
  по окончании, по приезде, по прилете
\item
  Наречия
  \begin{enumerate}
  \item
    Если наречное слово сохраняет связь с падежом существительного (в насмешку, с насмешкой, в шутку)
  \item
    Есть предлог без (без понятия, без устали, без толку, без оглядки)
  \item
    Есть предлоги в, на + суффиксы ах, ях (в ночах, на часах)
  \item
    Наречия + предлоги + наречия (шаг за шагов, бок о бок, точка в точку)
  \item
    Предлог в + гласная (в открытую, в упор, в обрез)
  \end{enumerate}
\item
  Союзы: потому что, так как, как будто
\end{itemize}

\newpage
\noindent\makebox[\linewidth]{\rule{\paperwidth}{0.4pt}}
\section{(05.04.2019)}
\noindent\makebox[\linewidth]{\rule{\paperwidth}{0.4pt}}

\subsection{Задание 21}

Двоеточие

\begin{enumerate}
\item
  Простое предложение с обобщающим словом перед однородными членами. О: о, о, о.
  В саду росли разные \it{цветы}: розы, гвоздики, пионы.
\item
  Простое предложение без обобщающего слова. На собрании присутствовали: Иванов, Петров, Сидоров.
\item
  При цитировании и прямой речи. Базаров говорил: "Мой дед землю пахал".
\item
  Сложное бессоюзное предложение.
  \begin{itemize}
  \item
    2-е поясняет, раскрывает 1-е (а именно, то есть).
    Долина пестреет разными цветами: желтеет дрок, синеют колокольчики, белеют ромашки.
  \item
    Если 2-е предложение содержит причину.
    Любите книгу: она помогает вам разобраться в жизненных ситуациях.
  \item
    2-е разъясняет 1-е
    Я поднял голову: в небе летел клин журавлей.
  \end{itemize}
\end{enumerate}

Тире

\begin{enumerate}
\item
  Простое предложение с обобщающим словом после однородных. О, о, о - О.\\
  Розы, тюльпаны, гвоздики --- разные цветы росли в саду.
\item
  Между подлежащим и сказуемым.
\item
  Неполное предложение.\\
  Я за свечку --- свечка в печку.
\item
  Сложносочиненные (неожиданные поворот).\\
  Минута --- и стехи свободно потекут.
\item
  Сложное бессоюзное предложение
  \begin{itemize}
  \item
    Быстрая смена событий.\\
    Сыр выпал --- с ним была плутовка такова.
  \item
    Противопоставление.\\
    Ввысь взлетел Сокол --- жмётся Уж к земле.
  \item
    Можно восстановить придаточное времени (когда, если, то).\\
    Лес рубят --- щепки летят.
  \item
    Если есть сравнение (как, словно, точно)\\
    Молвит слово --- соловей поёт.
  \item
    2-я часть --- вывод\\
    Не было возможности выйти тайно --- он вышел открыто.
  \end{itemize}
\item
  Приложение
  \begin{itemize}
  \item
    Приложение имеет уточняющее значение (а именно)\\
    На август было назначено интересное спортивное мероприятие --- бег по пересеченной местности.
  \item
    Необходимо установить грань между приложением и определяемым словом\\
    Лютейший бич небес, природы ужас --- мор свирепствует в лесах.
  \item
    Чтобы отделить приложения от однородных членов.\\
    На террасе я увидел бабушку, Ивана Кузьмича --- соседа по квартире, сестру с подругой.
  \end{itemize}
\item
  При цитировании, прямой речи, если слова автора стоят после цитаты\\
\item
  Для обозначения диапазона\\
  На линии Казань --- Москва ходят теплоходы.
\end{enumerate}

Запятая

\begin{enumerate}
\item
  При однородных членах.\\
  Одна запятая:
  \begin{itemize}
  \item
    2 однородных члена\\
    В саду росли розы, тюльпаны.
  \item
    3 однородных члена + 1 союз\\
    В саду росли розы, тюльпаны и гвоздики
  \item
    Две пары однородных\\
    В саду росли розы и тюльпаны, гвоздики и пионы.
  \item
    С противительными союзами (а, но, да=но, однако, же, зато)\\
    В саду росли не розы, а тюльпаны.
  \item
    С двойными сопоставительными союзами (не только, но и; не столько, сколько; так же, как и; хотя и, но; если не, то)\\
    В саду росли не только розы, но и тюльпаны.
  \item
    С повторяющимися союзами\\
    В саду росли не то розы, не то гвоздики.\\
    ЗАМЕЧАНИЕ: если повторяющийся союз входит в состав фразеологизма, то запятая не ставится.\\
    Ни пуха ни пера, ни рыба ни мясо
  \end{itemize}
  Две запятые:
  \begin{itemize}
  \item
    Три однородных члена
  \item
    3 однородных + 3 повторяющихся союза
  \item
    3 однородных + 2 союза\\
    В саду росли розы, и тюльпаны, и гвоздики.
  \item
    3 пары однородных
  \end{itemize}
\item
  С сочинительными предложениями\\
  Ставится (Лошади тронулись, и колокольчик зазвенел)\\
  Не ставится:
  \begin{itemize}
  \item
    Если есть общий второстепенный член (обстоятельство места, времени)\\
    В парке играл оркестр и танцевали пары.
  \item
    Если обе части односостаные безличные.\\
    Подморозило и лужицу затянуло льдом.
  \item
    Есть общая побудительная интонация\\
    Как прекрасен этот мир и хорошо жить на свете.
  \end{itemize}
\item
  При вставных, вводных конструкциях.\\
\item
  Сложное бессоюзное предложение\\
  Если действие происходит одновременно или последовательно.\\
  Пример: Лошади тронулись, колокольчик загремел, кибитка понеслась.
\item
  Обращение\\
  О не выделяется запятыми, присоединяется к обращению\\
  Пример: Как хорошо ты, о море ночное!\\
  Пример: Шагай, страна, быстрей, моя
\item
  Вводные слова и конструкции\\
  Не являются членами предложения\\
  \begin{itemize}
  \item
    Отношение говорящего к содержанию высказывания\\
    \begin{itemize}
    \item
      Степень достоверности: конечно, несомненно, кажется, возможно, скорее всего\\
      Пример: Независимость собственных мнений, разумеется, дело почтенное и благое.
    \item
      Эмоциональная оценка: к счастью, к беде, к стыду, к сожалению\\
      Пример: Был в старину народ, к стыду земных племен, который до того в сердцах ожесточился, что противу богов
      вооружился.
    \end{itemize}
  \item
    Ссылка на источник: говорят, сообщают, по-моему, по чьему-то мнению.\\
    Пример: по мнению Белинского, Онегин и Печорин являются лишними людьми.
  \item
    Обычность сообщения: как правило, бывает, по обыкновению, случается\\
    Пример: Как правило, Базаров вставал рано и сразу принимался за дела.
  \item
    Оформление речи
    \begin{itemize}
    \item
      Последовательность мыслей: во-первых, например, следовательно\\
      Пример: Вот в Риме, например, я видел огурец.
    \item
      Оценка речи: грубо говоря, если можно так выразиться, так сказать\\
      Пример: Ломоносов, так сказать, сам был первым нашим университетом.
    \end{itemize}
  \item
    Привлечение внимания: представьте себе, извините, простите, помилуй, знаете ли\\
    Пример: Помилуй, мне и отроду нет году, - ягненок говорит.
  \end{itemize}
  \paragraph{Замечания} не являются вводными:
  авось, бишь, буквально, будто, в добавок, в довершение, вдруг, ведь, в то же время, в конечном счёте, вот, всё-таки,
  в принципе, вряд ли, исключительно, как раз, между тем, тем не менее, небось, никак, почти, поэтому, прежде всего,
  приблизительно, примерно, притом, причем, просто, решительно, словно, якобы\\
  Может быть:
  \begin{itemize}
  \item
    Вводное. Можно переставить\\
    Чацкий, может быть, ещё выстоит.
  \item
    Не вводное.\\
    Что может быть интереснее этой книги
  \end{itemize}
  Однако:
  \begin{itemize}
  \item
    Вводное. В середине или конце предложения
  \item
    Не вводное. В начале предложения или одной из частей ССП.
  \end{itemize}
  Наконец:
  \begin{itemize}
  \item
    Вводное. В конце предложения, после перечисления.\\
    Снег растаял, прилетели птицы и, наконец, появилась трава.
  \item
    Не вводное. В начале конструкции (наконец-то)\\
    Когда наконец выглянуло солнце, мы отправились на прогулку.
  \end{itemize}
  Кстати:
  \begin{itemize}
  \item
    Вводное. Вдобавок\\
    Кстати, ты сегодня придёшь?
  \item
    Не вводное. "к месту", "вовремя"\\
    Он сказал это кстати.
  \end{itemize}
\item
  Вводные (вставные) конструкции\\
  Попутные замечания, комментарии автора.
  \begin{itemize}
  \item
    Запятая\\
    Праведник Савелий, по логике автора, духовно связан с великими подвижниками.
  \item
    Тире\\
    Электрические сны на яву --- таким видится поэту синематограф --- пугают Блока как демоническая выдумка.
  \item
    Скобки\\
    Матрена видит причину гибели сына в том, что на Рождество она надела чистую рубашку (дурная примета!)
  \end{itemize}
  
\end{enumerate}

Про него уральцы говорят: "Тургояк --- младший брат Байкала". И не случайно: вода Тургояка по чистоте и прозрачности
близка к байкальской. Размеры озера тоже сравнимы с величиной старшего брата: его площадь составляет 2638 гектаров.
Близ озера раскинулся национальный парк Таганай, включающий несколько горных массивов: Таганай, Юрма и Ицыл.
Многие истории были когда-то мифами: о загадочном народе, проживавшем здесь, повествует древнеславянская и
финноугорская мифология.

Усть-ленский заповедник расположен в Якутии в зоне вечной мерзлоты --- царстве арктической природы.
Коренные жители считают, что это место --- обитель священных духов. Проходя мимо острова на лодке нужно бросить в воду
дань --- монетки. Около столба часто наблюдают миражи --- оптические эффекты. Общая площадь дельты реки Лены ---
крупнейшей реки Сибири --- составляет более 30000 кв. км.

Великолепие ... гольцов очаровывает человека, монументальные виды покоряют сердце и воображение.
Горы тянутся с запада на восток, закрывая прекрасную Тункинскую долину от суровых северных ветров.
Живописными пейзажами гольцы похожи на горы, за что их называют тункинскими Альпами.
Природа трудилась миллионы лет, чтобы сотворилось это природное чудо.

\end{document}
