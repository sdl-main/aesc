\documentclass{article}

\usepackage{titlesec}
\usepackage[T1,T2A]{fontenc}
\usepackage[utf8]{inputenc}
\usepackage[english,russian]{babel}

\author{SDL}

\titleformat{\section}
            {\normalfont\Large}
            {}
            {0pt}
            {Урок \thesection\quad}

\begin{document}
\noindent\makebox[\textwidth]{\rule{\paperwidth}{0.4pt}}
\section{(07.09.2018)}
\noindent\makebox[\textwidth]{\rule{\paperwidth}{0.4pt}}

\subsection{Сборники тестов по русскому}
\begin{itemize}
\item
  Цыбулько - Подготовка к ЕГЭ по русскому языку
\item
  Егораева - Подготовка к ЕГЭ, Сборник заданий
\item
  Сенина - Подготовка к ЕГЭ
\end{itemize}

\paragraph{Орфографические задания} включают в себя задания с 9 по 15.

\subsection{Задание №9. Правописание гласных в корне}

\begin{enumerate}
\item
  Проверяемая гласная в корне. (ронять - урон, цветение - цвет)
\item
  Непроверяемая гласная в корне. (бетон, фантазии)
\item
  Чередующаяся гласная в корне.
\end{enumerate}

\paragraph{Чередующаяся гласная в корне}
\begin{enumerate}
\item
  \textbf{Чет-чит, бер-бир, мер-мир, пер-пир, стел-стил, блест-блист, жег-жиг, дер-дир, тер-тир, ня-ни, ча-чи.}
  Если после корня \^{а}, пишем ``и''. (замереть - замирать, отпереть - отпирать, замять - заминать).
  Исключение: сочетать - сочетание.
\item
  \textbf{Лож - лаг, кос - кас.} Если есть \^{а}, пишем ``а''. (положить - полагать, коснуться - касаться).
  Исключение: п\'{о}лог.
\item
  \textbf{Клан - клон, твар - твор, гар - гор.} О - в безударных корнях. (тварь - творчество - сотворить, кланяться -
  поклониться, загар - загореть). Исключения: \'{у}тварь, выгарки, пригарь.
\item
  \textbf{Зор - зар.} ``а'' в безударном положении. (зори - зарево - озарение). Исключение: зорянка, заревать.
\item
  \textbf{Плав - плов - плыв.} ``о''/''ы'' только в исключениях. (поплавок - плавучий).
  Исключения: пловец, пловчиха, плывун.
\item
  \textbf{Рост - раст - ращ.} Перед ``ст'', ``щ'' - а, в остальных случаях - о. (возраст - выращенный, водоросль).
  Исключения: Ростов, росток, Ростислав, ростовщик, отрасль
\item
  \textbf{Скач - скак - скоч.} ``Скач'' и ``скак'' обозначают многократное действие, ``скоч'' обозначает однократное..
  (проскакать - выскочить). Исключения: скачок, скачи
\item
  \textbf{Мак - мок.} ``Мак'' имеет значение ``опускать в жидкость'',
  ``мок'' - ``пропускать жидкость''. (обмакнуть кисть, непромокаемый плащ)
\item
  \textbf{Равн - ровн.} ``Равн'' в словах со значением ``одинаковый'', ``сходный''. ``Ровн'' в значении ``гладкий'',
  ``прямой''. (уравнение, подровнять). Исключения: уровень, ровестник, ровня, поровну, равнина.
\end{enumerate}

\noindent\makebox[\linewidth]{\rule{\paperwidth}{0.4pt}}
\section{(14.09.18)}
\noindent\makebox[\linewidth]{\rule{\paperwidth}{0.4pt}}
\paragraph{}

Не надо путать омонимичные корни:

\begin{enumerate}
\item
  \textbf{мер} - из\textbf{мер}яемый
\item
  \textbf{мир} - с\textbf{мир}ение
\item
  \textbf{кос} - по\textbf{кос}ился
\item
  \textbf{гор} - \textbf{гор}евать
\end{enumerate}

\paragraph{Словарный диктант (проверяемые гласные)}

Уплотнить сроки, 
приласкать собаку, 
обнажить пороки, 
раскалить сковороду, 
усложнить обстановку, 
облокатиться о перила, 
отдалить поездку, 
поглощать энергию, 
отказаться от услуг, 
угрожать расправой,  
оправдать поступок, 
вопиющий произвол, 
юный запивала, 
зарядить ружье,
роптать на судьбу,
смягчение приговора,
увядание красоты,
истинное заглядение,
высокое заграждение,
зачарованный взгляд,
обрамление картины,
нагревание предмета,
лечебное голодание,
неподдельное удивление,
чистосердечное покаяние,
вечное смирение,
внезапное озлобление,
невыносимое угнетение,
далёкая сторона,
упрощенный пример

\begin{itemize}
\item
  \textbf{Полногласие}: оло, оро, еле, ере. (молоко, корова, берег, ворота)
\item
  \textbf{Неполногласие}: ра, ле, ре. (млеко, брег, врата)
\end{itemize}

\paragraph{}

Если дано слово с полногласием, его надо проверять словом с полногласием.
\paragraph{}
Если слово с неполногласием, его надо проверять словом с неполногласием

\paragraph{Словарный диктант (чередующаяся гласная)}
Блистательный, блестящий, обжигание, пробираться, расстелить, расстилаться, сочетать, касаться, неукоснительно,
предложить, безотлагательный, зарница, озарять, погорелец, сотворение, отклонение, плавучий, пловец, вскочить,
обскакать, нарастание, произрастать, отскочить, взрослеть, наклонять, Ростислав, излагать, обложение, замереть,
отпирать, растворить, утварь

\paragraph{Найти слово с проверяемой безударной гласной в корне}

Замереть, поплавок, ветеран, росток, \underline{примирить врагов}

\end{document}
