\documentclass{article}

\usepackage{titlesec}
\usepackage{subcaption}
\usepackage{longtable}
\usepackage{booktabs}
\usepackage[T1,T2A]{fontenc}
\usepackage[utf8]{inputenc}
\usepackage[english,russian]{babel}

\author{SDL}

\titleformat{\section}
            {\normalfont\Large}
            {}
            {0pt}
            {Урок \thesection\quad}

\begin{document}
\noindent\makebox[\textwidth]{\rule{\paperwidth}{0.4pt}}
\section{(07.09.2018)}
\noindent\makebox[\textwidth]{\rule{\paperwidth}{0.4pt}}

\subsection{Сборники тестов по русскому}
\begin{itemize}
\item
  Цыбулько - Подготовка к ЕГЭ по русскому языку
\item
  Егораева - Подготовка к ЕГЭ, Сборник заданий
\item
  Сенина - Подготовка к ЕГЭ
\end{itemize}

\paragraph{Орфографические задания} включают в себя задания с 9 по 15.

\subsection{Задание №9. Правописание гласных в корне}

\begin{enumerate}
\item
  Проверяемая гласная в корне. (ронять - урон, цветение - цвет)
\item
  Непроверяемая гласная в корне. (бетон, фантазии)
\item
  Чередующаяся гласная в корне.
\end{enumerate}

\paragraph{Чередующаяся гласная в корне}
\begin{enumerate}
\item
  \textbf{Чет-чит, бер-бир, мер-мир, пер-пир, стел-стил, блест-блист, жег-жиг, дер-дир, тер-тир, ня-ни, ча-чи.}
  Если после корня \^{а}, пишем ``и''. (замереть - замирать, отпереть - отпирать, замять - заминать).
  Исключение: сочетать - сочетание.
\item
  \textbf{Лож - лаг, кос - кас.} Если есть \^{а}, пишем ``а''. (положить - полагать, коснуться - касаться).
  Исключение: п\'{о}лог.
\item
  \textbf{Клан - клон, твар - твор, гар - гор.} О - в безударных корнях. (тварь - творчество - сотворить, кланяться -
  поклониться, загар - загореть). Исключения: \'{у}тварь, выгарки, пригарь.
\item
  \textbf{Зор - зар.} ``а'' в безударном положении. (зори - зарево - озарение). Исключение: зорянка, заревать.
\item
  \textbf{Плав - плов - плыв.} ``о''/''ы'' только в исключениях. (поплавок - плавучий).
  Исключения: пловец, пловчиха, плывун.
\item
  \textbf{Рост - раст - ращ.} Перед ``ст'', ``щ'' - а, в остальных случаях - о. (возраст - выращенный, водоросль).
  Исключения: Ростов, росток, Ростислав, ростовщик, отрасль
\item
  \textbf{Скач - скак - скоч.} ``Скач'' и ``скак'' обозначают многократное действие, ``скоч'' обозначает однократное..
  (проскакать - выскочить). Исключения: скачок, скачи
\item
  \textbf{Мак - мок.} ``Мак'' имеет значение ``опускать в жидкость'',
  ``мок'' - ``пропускать жидкость''. (обмакнуть кисть, непромокаемый плащ)
\item
  \textbf{Равн - ровн.} ``Равн'' в словах со значением ``одинаковый'', ``сходный''. ``Ровн'' в значении ``гладкий'',
  ``прямой''. (уравнение, подровнять). Исключения: уровень, ровестник, ровня, поровну, равнина.
\end{enumerate}

\noindent\makebox[\linewidth]{\rule{\paperwidth}{0.4pt}}
\section{(14.09.18)}
\noindent\makebox[\linewidth]{\rule{\paperwidth}{0.4pt}}
\paragraph{}

Не надо путать омонимичные корни:

\begin{enumerate}
\item
  \textbf{мер} - из\textbf{мер}яемый
\item
  \textbf{мир} - с\textbf{мир}ение
\item
  \textbf{кос} - по\textbf{кос}ился
\item
  \textbf{гор} - \textbf{гор}евать
\end{enumerate}

\paragraph{Словарный диктант (проверяемые гласные)}

Уплотнить сроки, 
приласкать собаку, 
обнажить пороки, 
раскалить сковороду, 
усложнить обстановку, 
облокатиться о перила, 
отдалить поездку, 
поглощать энергию, 
отказаться от услуг, 
угрожать расправой,  
оправдать поступок, 
вопиющий произвол, 
юный запивала, 
зарядить ружье,
роптать на судьбу,
смягчение приговора,
увядание красоты,
истинное заглядение,
высокое заграждение,
зачарованный взгляд,
обрамление картины,
нагревание предмета,
лечебное голодание,
неподдельное удивление,
чистосердечное покаяние,
вечное смирение,
внезапное озлобление,
невыносимое угнетение,
далёкая сторона,
упрощенный пример

\begin{itemize}
\item
  \textbf{Полногласие}: оло, оро, еле, ере. (молоко, корова, берег, ворота)
\item
  \textbf{Неполногласие}: ра, ле, ре. (млеко, брег, врата)
\end{itemize}

\paragraph{}

Если дано слово с полногласием, его надо проверять словом с полногласием.
\paragraph{}
Если слово с неполногласием, его надо проверять словом с неполногласием

\paragraph{Словарный диктант (чередующаяся гласная)}
Блистательный, блестящий, обжигание, пробираться, расстелить, расстилаться, сочетать, касаться, неукоснительно,
предложить, безотлагательный, зарница, озарять, погорелец, сотворение, отклонение, плавучий, пловец, вскочить,
обскакать, нарастание, произрастать, отскочить, взрослеть, наклонять, Ростислав, излагать, обложение, замереть,
отпирать, растворить, утварь

\paragraph{Найти слово с проверяемой безударной гласной в корне}

Замереть, поплавок, ветеран, росток, \underline{примирить врагов}




\paragraph{}
\noindent\makebox[\linewidth]{\rule{\paperwidth}{0.4pt}}
\section{(21.09.18)}
\noindent\makebox[\linewidth]{\rule{\paperwidth}{0.4pt}}

\paragraph{Задание 11.} Правописание приставок. Правописание и/ы после приставок.
  Разделительный мягкий и твердый знаки

\paragraph{Приставки на з/c - без, воз, вз, из, низ, раз, чрез, через}
Если после приставки следует глухой согласный - пишем с.\\
Пример: ра\underline{с}свет, и\underline{с}чезать, ни\underline{с}провергать, чере\underline{с}чур\\

Если после приставки следует звонкий согласный - пишем з.\\
Пример: ра\underline{з}бить, и\underline{з}дать, ни\underline{з}вергать, чре\underline{з}вычайный\\

! приставки з не существует (\underline{с}дать, \underline{с}шить)\\

В словах здесь, здоровье, здание, зга буква з является частью корня.\\

\paragraph{Приставки раз(с) - роз(с)}
Рассказ - россказни,
расписать - роспись, 
разлить - розлив, 
развалить - розвальни

\paragraph{Неизменяемые приставки над, об, от, под, пред, в}
Отдел, отчёт, обман, обсыпать

\paragraph{Приставка пра со значением "предок, прошлое"}
Пр\underline{а}бабушка, пр\underline{а}язык. НО пр\underline{о}образ

\paragraph{Правописание букв ы/и после приставок}
ы пишется после русских твёрдых приставок (предыстория, разыскать, отыграть)\\
и пишется после русских приставок меж и сверх (межинститутский, сверхизысканный)\\
и пишется после иностранных приставок пан, суб, пост, дез, контр, транс (постимпрессионизм)\\
и пишется в сложносокращенных словах (самиздат, пединститут)\\
Запомнить! взимать, но изымать

\paragraph{Словарный диктант (с/з)}

Восхождение на вершину, расчётный счёт, чересчур много, восполнить пропущенное,
безграничное доверие, пламенное воззвание, расформировать полк, истратить деньги,
чрезмерный восторг, искоренить недостатки, иссякший источник, бесперспективный план,
ниспровергать авторитеты, разжечь костёр, безветренный вечер, бесформенная масса,
бесжалостный поступок, бессмертное произведение, бессрочный отпуск, воспитать ученика,
воспроизвести текст, возрастной ценз, восседать на престоле, резкий тон, сдвинутые шкафы,
сдержать слово, низшая порода, распаковать вещи.

\paragraph{Словарный диктант (ы/и)}

Безыскусственный, небезызвестный, предыстория, сверхизысканный, безысходный,
межинститутский, взимать, изымать, субинспектор, трансиорданский, возыметь,
обыскать, дезинформация, обындеветь, подытожить, постимпрессионизм, сверхиндустриализация,
спортигра, предыюльский, сызмала, отымённый, панисламизм, фининспектор, межигровой.

\newpage
\noindent\makebox[\linewidth]{\rule{\paperwidth}{0.4pt}}
\section{(28.09.18)}
\noindent\makebox[\linewidth]{\rule{\paperwidth}{0.4pt}}

\subsection{Приставки пре- и при-}
\paragraph{}

\textbf{Пре-}
\begin{enumerate}
\item
  Когда обозначает "очень, высокая степень чего-либо". Примеры: пр\underline{е}возносить,
  пр\underline{е}увеличивать, пр\underline{е}мудрый.
\item
  Когда можно заменить приставкой пере-. Примеры: пр\underline{е}дание, пр\underline{е}ступление,
  пр\underline{е}вратник, пр\underline{е}емник.
\end{enumerate}
\paragraph{}

\textbf{При-}
\begin{enumerate}
\item
  Значение близости, присоединения, начала. Примеры: пр\underline{и}станционный, пр\underline{и}ступить,
  пр\underline{и}писать, Пр\underline{и}морье.
\item
  Неполнота действия (можно подставить наречие "чуть-чуть"). Примеры: пр\underline{и}открыть,
  пр\underline{и}спустить, пр\underline{и}утихнуть.
\item
  Доведение действия до предела. Примеры: пр\underline{и}искать, пр\underline{и}выкнуть,
  пр\underline{и}учить, пр\underline{и}ручить.
\item
  Действие в собственных интересах. Примеры: пр\underline{и}манить, пр\underline{и}нарядиться,
  пр\underline{и}слушаться (к совету), пр\underline{и}своить.
\item
  Сопутствующее действие. Примеры: пр\underline{и}прыгивать, пр\underline{и}свистывать,
  пр\underline{и}танцовывать.
\end{enumerate}
\paragraph{}

\paragraph{Русские словарные слова}

\paragraph{Пре-} Пр\underline{е}словутый, пр\underline{е}имущество, непр\underline{е}минуть,
пр\underline{е}пинание, пр\underline{е}рекаться, пр\underline{е}кословить, пр\underline{е}ткновение,
пр\underline{е}тить.

\paragraph{При-} Без пр\underline{и}крас, непр\underline{и}ступный, пр\underline{и}скорбный,
пр\underline{и}вередлиный, пр\underline{и}гожий, пр\underline{и}личный, пр\underline{и}язнь,
пр\underline{и}тязание, воспр\underline{и}ятие, пр\underline{и}звание, пр\underline{и}сутствовать,
воспр\underline{и}емник.

\paragraph{Иностранные словарные слова}

\paragraph{Пре-} Пр\underline{е}амбула, пр\underline{е}валировать, пр\underline{е}зидент,
пр\underline{е}зумпция, пр\underline{е}мьера.

\paragraph{При-} Пр\underline{и}ватный, пр\underline{и}вилегии, пр\underline{и}митивный,
пр\underline{и}оритет.

\begin{longtable}[c]{|p{3cm}|p{3cm}|}
  \caption{Парные слова по контексту}\\
  \hline
  пр\underline{е}дать друга & пр\underline{и}дать форму шара\\
  \hline
  предел мечтаний & придел в церкви\\
  \hline
  Презрение к трусу & призрение сирот\\
  \hline
  преклоняться перед красотой & приклонить ветки\\
  \hline
  претворить планы в жизнь & притвориться больным\\
  \hline
  претерпеть изменения & притерпеться к запаху\\
  \hline
  преходящий (временный) & приходящая няня\\
  \hline
  непреходящий (вечный) & приходящая няня\\
  \hline
  пребывать в городе & прибывать в город\\
  \hline
\end{longtable}

\subsection{Разделительные ъ/ь}

\paragraph{}
\textbf{Ъ}

\begin{enumerate}
\item
  После твёрдых \textbf{приставок}. Примеры: об\underline{ъ}езд, кон\underline{ъ}юнктура,
  ад\underline{ъ}ютант.
\item
  После корней \textbf{двух}, \textbf{трёх}, \textbf{четырёх}. Примеры: двух\underline{ъ}ядерный.
\end{enumerate}

\paragraph{}
\textbf{Ь}
\begin{enumerate}
\item
  В корне слова. Примеры: в\underline{ь}юга, обез\underline{ь}яны, б\underline{ь}ют.
\item
  В притяжательных прилагательных перед и, е. Примеры: птич\underline{ь}и, помещич\underline{ь}ему.
\item
  В существительных множественного числа перед и, я, е. Примеры: сынов\underline{ь}я, руч\underline{ь}и.
\item
  В иноязычных словах на о. Примеры: буль\underline{о}н, почталь\underline{о}н, медаль\underline{о}н,
  ожерел\underline{ь}е.
\end{enumerate}

\paragraph{Запомнить слово: фельд\underline{ъ}егерь.}

\paragraph{Диктант на пре- при-} Прибывать в город, пребывать в неволе, преемник традиций, радиоприёмник,
предание старины глубокой, придать блеск поверхности, презирать негодяя, призреть бездомных детей, претворить
планы в жизнь, притворить дверь, камень преткновения, приткнуться в уголок, преходящий момент, приходящая няня,
претерпеть лишения, притерпеться к боли, предел мечтаний, придел в храме, преподать урок, припадать к плечу,
приклонить ветки, преклонить колени, пресветлый образ, прискорбный файт, преемлимый вариант,
приемственность поколений.

\noindent\makebox[\linewidth]{\rule{\paperwidth}{0.4pt}}
\section{(05.10.18)}
\noindent\makebox[\linewidth]{\rule{\paperwidth}{0.4pt}}

\paragraph{Диктант на разделительные Ъ и Ь} Партьер, пьеса, бульон, предъявить, изъян, обезьяна, съязвить,
лисья, разъяренный, эскадрилья, трёхэтажный, сверхъестественный, мы бьём, въявь, сыновья, сэкономить,
съездить, мелколесье, подъём, объявление, сагитировать, арьергард, объективный, вьюга, межъярусный,
пьедестал, предъюбилейный, бильярд, отъявленный, фортепиано, безъядерный, рьяный, интерьер, предъявитель.

\paragraph{}

Первые четыре задания: даётся микротекст.
\begin{enumerate}
\item
  Выбрать два предложения, называющие главную мысль текста.
\item
  Вставить слово в пропуск.
\item
  Угадать слово по контексту.
\end{enumerate}

\paragraph{Образец заданий 1-3.}
\begin{enumerate}
\item
Чаще всего преломление лучей света в воздухе незначительно, а вот ложка в стакане чая кажется нам сломанной.
\item
  Причина заключается в различной плотности воды и воздуха.
\item
  ... переходя из одной среды в другую, лучи света преломляются, изменяя прямолинейный путь, отклоняясь по закону
физики в сторону более плотной среды.
\end{enumerate}

Вставить в пропуск: наоборот, \underline{поэтому}, так как, зато, вследствие.\\

\paragraph{}

\subsection{Четвертое задание --- расстановка ударения. (см. индекс ударений)}

\paragraph{Дополнение:}
\begin{enumerate}
\item
  Глаголы прошедшего времени женского рода: несл\'{а}, везл\'{а}.
  \textbf{НО}: кл\'{а}ла, кр\'{а}ла, стл\'{а}ла, рж\'{а}ла.
\item
  Т\'{о}рты, п\'{о}рты, аэропо\'{р}ты.
\item
  Сложные слова с корнем \textit{лог}. Некрол\'{о}г, катал\'{о}г, диал\'{о}г, монол\'{о}г.
\item
  Сложные слова с корнем \textit{вод}. Мусоропров\'{о}д, газопров\'{о}д, водопров\'{о}д.
  \textbf{НО}: электропр\'{о}вод.
\end{enumerate}

\paragraph{} Примеры заданий: 

\begin{itemize}
\item
  Прож\'{и}вший, д\'{о}верху, позвал\'{а}, сл\'{и}вовый, одобрен\'{а}.
\item
  З\'{а}гнутый, дон\'{е}льзя, откл\'{ю}ченный, приб\'{ы}в, ч\'{е}люстей.
\item
  Облил\'{а}сь, аэроп\'{о}рты, нед\'{у}г, опт\'{о}вый.
\item
  Убыстр\'{и}ть, моза\'{и}чный, д\'{о}верху, с\'{о}гнутый, жалюз\'{и}.
\item
  Пр\'{и}нятый, св\'{ё}кла, облегч\'{и}т, к\'{у}хонный, кор\'{ы}сть.
\item
  Ср\'{е}дство, пон\'{я}вший, окруж\'{и}т, кр\'{а}ны, исч\'{е}рпав.
\item
  Бухг\'{а}лтеров, вероиспов\'{е}дование, повтор\'{е}нный, обостр\'{и}ть, сир\'{о}ты.
\item
  Катал\'{о}г, полож\'{и}л, рвал\'{а}, окл\'{е}ить, прин\'{у}дить.
\end{itemize}

\subsection{23 задание --- типы речи}

\begin{enumerate}
\item
  Описание (портрет, пейзаж, интерьер). Основные части речи --- прилагательное, наречие, причастие.
\item
  Повествование (раскадровка). Часть речи --- глагол.
\item
  Рассуждение (нельзя передать изображением). Абстрактные понятия, специальные конструкции, наличие вводных слов,
  риторические восклицания, обращения, вопросы, сложноподчиненные предложения с придаточными причины, следствия,
  уступки, цели. Слова потому что, так что, хотя, чтобы.
\end{enumerate}

\end{document}
