\documentclass[dvipdfmx]{article}

\usepackage{titlesec}
\usepackage[T1,T2A]{fontenc}
\usepackage[utf8]{inputenc}
\usepackage[english,russian]{babel}

\titleformat{\section}
            {\normalfont\Large\bfseries}
            {}
            {0pt}
            {Урок \thesection\quad}

\begin{document}

\noindent\makebox[\linewidth]{\rule{\paperwidth}{0.4pt}}
\section{(17.09.18)}
\noindent\makebox[\linewidth]{\rule{\paperwidth}{0.4pt}}

\subsection{Первая мировая война}
\paragraph{}
\textbf{Тройственный союз}: Германия, Австро-Венгрия, Италия
\paragraph{}
\textbf{Антанта}: Россия, Великобритания, Франция.
\paragraph{}

В 1914 году в Антанту входит Япония. Единственное, что она сделает - её войска оккупируют германские колонии.

\paragraph{}

\textbf{Причины войны:}
Резкое обострение противоречий между импералистическими государствами в связи с борьбой за колонии.

Инициатор войны - Германия и другие страны, которые хотели расширить свои владения.

Резкие внутренние противоречия в странах. Буржуазия приобрела необычайную мощь, давление на крестьян и рабочих нарастало.
С переходом к империализму усилилась эксплуатация населения.

В 1905-1907 произошла первая революция в России. В итоге в России в 1906 году появился парламент, госдума.

К 1914 году во всех европейских странах сложились революционные настроения.

Ленин придумал идею свержения правительства - проиграть в первой мировой войне, чтобы недовольство народа ослабило власть.

\subsection{Цели войны}

\paragraph{Тройственный союз}
\begin{itemize}
\item
  Германия хотела расширить свои владения в Европе за счёт Польши, Франции, Бельгии и России.
  Передел колониальных империй, для чего надо было заставить Францию и Россию капитулировать.
\item
  Австо-Венгрия претендовала на юг России.
\end{itemize}
\paragraph{Антанта}
\begin{itemize}
\item
  Великобритании не нравилась Германская империя, которая грозила могуществу и славе великой морской державы.
\item
  Франции нужен был возврат Эльзасских и Лотарингских земель, претензии на германский Саар
\item
  Россия хотела распространить влияние на Средиземноморские проливы, имела претензии на польские территории, желала
  возрождения Византийской империи (для этого нужен был Стамбул (Константинополь))
\end{itemize}

\newpage
\noindent\makebox[\linewidth]{\rule{\paperwidth}{0.4pt}}
\section{(24.09.18)}
\noindent\makebox[\linewidth]{\rule{\paperwidth}{0.4pt}}
\paragraph{}

Вильгельм II, правитель Германии, затеял эту войну. К 1914 году Германия была единственной страной в мире,
полностью готовой к войне.

\subsection{Повод к войне}
Австро-Венгрия привела свои войска с наследником престола Францем Фердинандом к границе Сербии.
Сербии это не понравилось, и 28 июня 1914 года Франца Фердинанда убил Гаврило Принцип.
Австро-Венгрия предъявила Сербии ультиматум, в котором было требование допустить до расследования
инцидента австрийских чиновников, которое Сербия отклонила. Австрии это не понравилось, и она объявила
войну Сербии. Это послужило толчком к началу Первой мировой войны.

1 августа 1914 года была объявлена война России.

3 августа 1914 года Франция объявила войну Австро-Венгрии и Германии.

\paragraph{}
\textbf{План Шлиффена-Мольтке} --- блицкриг.
По этому плану первый удар Германия наносит удар Франции через нейтральную Бельгию.

\paragraph{}

План провалился из-за:
\begin{enumerate}
\item
  4 августа немцы вторглись в Бельгию и тут же Англия объявила войну и перебросила свои войска во Францию.
\item
  Россия нанесла большой удар, увеличив фронт.
\end{enumerate}

\newpage
\noindent\makebox[\linewidth]{\rule{\paperwidth}{0.4pt}}
\section{(31.09.18)}
\noindent\makebox[\linewidth]{\rule{\paperwidth}{0.4pt}}

\paragraph{}
Под Парижем немецкую армию чуть не окружили, они начали стремительное отступление к реки Эна.
В декабре Турция начала наступление на Кавказе, но встретила неожиданный отпор России. Русские войска не только
вытолкнули турецкую армию со своей земли, но ещё и зашли на турецкую землю.

\paragraph{}
В 1915 году план Шлиффена был перевернут, теперь главный удар ведётся по России. В мае Германия выбросила Россию с
территории Польши. В ходе наступления германских и австро-венгерских войск Россия потерпела огромные потери.
Офицерский корпус в 1915 году изменился в классовом отношении, потому что в офицерские школы стали брать всех.

\paragraph{}
В сентябре Россия собралась с силами и дала отпор.

\paragraph{}
На 1916 год Германия запланировала грандиозное наступление на Францию под крепостью Верден.
1916 год --- самый кровопролитный год войны.

\paragraph{}
26 мая Николай II отдал приказ в Петроград прекратить беспорядки.
27 мая Петроград оказался в руках революционеров.
Госдума заставила императора подписать отречение от престола. В отречении он отрекся сам и за своего сына, чтобы нарушить
правила. Отречение признали и передали власть брату Михаилу.
2 марта 1917 года монархия рухнула. Россия по факту стала республикой. Распущенная госдума создала Временное правительство
во главе с князем Львовым. В его состав не вошли большевики, которые не признавали такую власть.
Все старые порядки царской армии отменялись. Вся полнота правления армией передалась Советам, в том числе советам рабочих
депутатов.

\paragraph{}
В России возникает двоевластие. Одна власть --- Временное правительство. Другая власть --- Советы. 
Временное правительство было властью, но не опиралось на силу, т.к. армия была в руках Советов.

\paragraph{}
Революция в России была буржуазно-демократической. Одна партия надеялась перевести её сразу в социалистическую,
но находились они заграницей и не могли попасть в Россию.
В апреле Ленину и его соратникам Германия позволила через свою территорию попасть в Россию.
Один из его противников --- Троцкий --- перешёл в партию большевиков. Он возглавил Петроградский совет, основал ВРК -
военно-революционный комитет.

\newpage
\noindent\makebox[\linewidth]{\rule{\paperwidth}{0.4pt}}
\section{(15.10.18)}
\noindent\makebox[\linewidth]{\rule{\paperwidth}{0.4pt}}


%HW: таблица по первой мировой войне
%Даты        Военные события на западном фронте      Военные события на восточной фронте     Важнейшие политические события
%1.8.1914
%...
%11.11.1918

\paragraph{}
Ленин: надо дать правительству издавать глупые законы, тогда его авторитет падёт и большевики придут к власти.
В июле в Петрограде разразилось вооруженное восстание против Временного правительства.
У военных не получилось, потому что эту
демонстрацию власть расстреляла. Большевики ушли в подполье. Ленин был вынужден скрываться.
В сентябре Ленин вернулся, чтобы совершить госпереворот.

\newpage
\noindent\makebox[\linewidth]{\rule{\paperwidth}{0.4pt}}
\section{(22.10.18)}
\noindent\makebox[\linewidth]{\rule{\paperwidth}{0.4pt}}

В концу 1916 года стало ясно, что Германия проигрывает.

В июне 1917 года битва под Львовом. Поражение России, второй кризис Временного правительства.
В апреле 1917 года США вступили в войну. Они испугались революций в России и того факта, что она может выйти из Антанты.
Чтобы не допустить проигрыша Антанты в войне, США вступили в Антанту.

Первые декреты советской власти:

\begin{enumerate}
\item
  Декрет о власти. Согласно нему все старые органы государственной власти ликвидировались. Отныне везде были Советы.
  Высший орган законодательной власти - Съезд Советов. Появился ВЦИК. Исполнительная власть - СНК.
\item
  Декрет о земле. Все земли у помещиков изымаются. Все они сводятся в общегосударственный фонд. Эти земли распределялась
  среди крестьян.
  Но эта земля вручалась на правах владения, но не собственности (нельзя было продавать, обменивать и т.д.).
  Земля находилась в бессрочном пользовании крестьян с правом передачи по наследству.  
\end{enumerate}

Россия вышла из войны.
Антанта войну выиграла, но Россия проиграла.


\end{document}
