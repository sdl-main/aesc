\documentclass[dvipdfmx]{article}

\usepackage{amsmath}
\usepackage{subcaption}
\usepackage{graphicx}
\usepackage{titlesec}
\usepackage{tikz}
\usepackage{caption}
\usepackage{float}
\usepackage{circuitikz}
\usepackage[T1,T2A]{fontenc}
\usepackage[utf8]{inputenc}
\usepackage[english,russian]{babel}

\titleformat{\section}
            {\normalfont\Large\bfseries}
            {}
            {0pt}
            {Урок \thesection\quad}
            
\newcommand{\rulesep}{\unskip\ \vrule\ }
\usetikzlibrary{circuits.ee.IEC}
\tikzset{circuit declare symbol = ac source}
\tikzset{set ac source graphic = ac source IEC graphic}
\tikzset{
         ac source IEC graphic/.style=
          {
           transform shape,
           circuit symbol lines,
           circuit symbol size = width 3 height 3,
           shape=generic circle IEC,
           /pgf/generic circle IEC/before background=
            {
             \pgfpathmoveto{\pgfpoint{-0.8pt}{0pt}}
             \pgfpathsine{\pgfpoint{0.4pt}{0.4pt}}
             \pgfpathcosine{\pgfpoint{0.4pt}{-0.4pt}}
             \pgfpathsine{\pgfpoint{0.4pt}{-0.4pt}}
             \pgfpathcosine{\pgfpoint{0.4pt}{0.4pt}}
             \pgfusepathqstroke
            }
          }
        }

\begin{document}

\noindent\makebox[\textwidth]{\rule{\paperwidth}{0.4pt}}
\section{(08.09.2018)}
\noindent\makebox[\textwidth]{\rule{\paperwidth}{0.4pt}}

\subsection{Магнитное поле}

\paragraph{}

Магнитное поле порождается движущимися электрическими зарядами (током).

\paragraph{}

\textbf{Индукция магнитного поля $\vec{B}$} - векторная величина, являющаяся силовой характеристикой магнитного поля.
Определяет, с какой силой поле $\vec{F}$ действует на заряд $q$, движущийся со скоростью $\vec{v}$.

\begin{equation*}
  \vec{F} = q[\vec{v} \times \vec{B}], \quad\quad \vec{B} = [\textup{Тл}]
\end{equation*}
\paragraph{}

Пусть мы переходим из одной системы отсчёта в другую.
Из преобразований Лоренца следует, что:

\begin{equation*}
  F_1 = F_0\sqrt{1-\frac{\upsilon^2}{c^2}},
\end{equation*}
\paragraph{}

где $F_0$ - сила в покое, $\upsilon$ - скорость системы отсчета.

\paragraph{}

Пусть два заряда покоятся. По закону Кулона

\begin{equation*}
  F_0 = \frac{1}{4\pi\varepsilon_0}\frac{q_1q_2}{r^2}
\end{equation*}
\paragraph{}

Перейдем в систему отсчета, движущуюся со скоростью $\upsilon$.\\ Найдем \textbf{обобщенную силу Лоренца}:

\begin{equation*}
  F_1 = \frac{1}{4\pi\varepsilon_0}\frac{q_1q_2}{r^2}
  \frac{\big( \sqrt{1-\frac{\upsilon^2}{c^2}} \big)^2}{\sqrt{1-\frac{\upsilon^2}{c^2}}} =
  \underbrace{\frac{1}{4\pi\varepsilon_0r^2}\frac{q_1q_2}{\sqrt{1-\frac{\upsilon^2}{c^2}}}}_{F_\textup{электр.}} -
  \underbrace{\frac{\upsilon^2}{c^2}\frac{q_1q_2}{\sqrt{1-\frac{\upsilon^2}{c^2}}}}_{F_\textup{магн.}}
\end{equation*}

\paragraph{}

Перепишем формулу для силы магнитного взаимодействия:

\begin{equation*}
  F_M = \frac{\upsilon q_1}{4\pi\varepsilon_0r^2c^2}\frac{\upsilon q_2}{\sqrt{1-\frac{\upsilon^2}{c^2}}}
\end{equation*}
\paragraph{}

Введём индукцию магнитного поля B

\begin{equation*}
  B = \frac{\upsilon q_2}{4\pi\varepsilon_0c^2r^2\sqrt{1-\frac{\upsilon^2}{c^2}}}
\end{equation*}
\paragraph{}
Тогда формула силы магнитного взаимодействия запишется следующим образом:

\begin{equation*}
  F_M = q_1\upsilon B
\end{equation*}
\paragraph{}

Её можно трактовать так: заряд $q_2$ создаёт поле и действует на заряд $q_1$ с силой $F_M$.

\paragraph{}

Для удобства введём константу $\mu_0$:

\begin{equation*}
  \mu_0 = \frac{1}{\varepsilon_0c^2}
\end{equation*}
\paragraph{}

Тогда формула для магнитной индукции

\begin{equation*}
  B = \frac{\mu_0}{4\pi}\frac{\upsilon q}{r^2\sqrt{1-\frac{\upsilon^2}{c^2}}}
\end{equation*}
\paragraph{}

\textbf{Сила Ампера} --- сумма сил Лоренца от нескольких зарядов

\begin{equation*}
  F_A = BIL
\end{equation*}

\newpage

\paragraph{}
Пусть есть бесконечный заряженный провод и заряд $Q$ на расстоянии $x$ от провода. $S$ --- площадь
сечения провода, $\rho$ --- объемная плотность заряда.
\paragraph{}
\noindent\makebox[\textwidth][c]{
\begin{minipage}{0.6\textwidth}
  \includegraphics[width=\linewidth,natwidth=461,natheight=316]{images/1.jpg}
\end{minipage}}
\paragraph{}

\begin{equation*}
  dF_\parallel = dFsin\alpha, \quad dF_\perp = dFcos\alpha, \quad dq = \rho Sdx
\end{equation*}

\begin{equation*}
  dF = \frac{1}{4\pi\varepsilon_0}\frac{Q\rho Sdx}{r^2}, \quad r = \frac{x}{cos\alpha},
  \quad dx = \frac{rd\alpha}{cos\alpha}
\end{equation*}

\begin{equation*}
  dF_\perp = \frac{1}{4\pi\varepsilon_0}\frac{Q\rho Sdx}{x^2}cos^2\alpha cos\alpha
\end{equation*}
\paragraph{}

Подставим $r$, $dx$ и проинтегрируем:

\begin{equation*}
  F_\perp = \int_{-\frac{\pi}{2}}^{\frac{\pi}{2}}\frac{1}{4\pi\varepsilon_0}\frac{Q\rho Scos\alpha}{x}d\alpha =
  \frac{Q\rho S}{2\pi\varepsilon_0x}
\end{equation*}
\paragraph{}

Перейдем в систему отсчёта, движущуюся вправо со скоростью $\upsilon$:

\begin{equation*}
  F' = F_0\sqrt{1-\frac{\upsilon^2}{c^2}}, \quad F_M = -\frac{\upsilon^2}{c^2}\frac{Q\rho S}
  {2\pi\varepsilon_0x\sqrt{1-\frac{\upsilon^2}{c^2}}}
\end{equation*}

\newpage

Рассмотрим ток в проводнике с поперечным сечением $S$. Пусть средняя скорость электронов $u$,
$n$ --- объемная концентрация электронов.

\paragraph{}
\noindent\makebox[\textwidth][c]{
  \begin{minipage}{0.6\textwidth}
    \includegraphics[width=\linewidth,natwidth=699,natheight=185]{images/2.jpg}
\end{minipage}}
\paragraph{}

\begin{equation*}
  I = \frac{\Delta q}{\Delta t} = \frac{enSu\Delta t}{\Delta t} = neSu
\end{equation*}
\paragraph{}

Перепишем формулу $F_M$:

\begin{equation*}
  F_M = -uQ\frac{\mu_0}{2\pi}\frac{\rho S\upsilon}{x\sqrt{1-\frac{\upsilon^2}{c^2}}}
\end{equation*}
\paragraph{}

Но так как $\rho S\upsilon = neS\upsilon = I$, имеем

\begin{equation*}
  F_M = -uQ\frac{\mu_0}{2\pi}\frac{I}{x\sqrt{1-\frac{\upsilon^2}{c^2}}}
\end{equation*}
\paragraph{}

Мы получили формулу \textbf{силы взаимодействия заряда и бесконечного провода.}

\paragraph{}

Рассмотрим теперь случай двух проводников

\noindent\makebox[\linewidth][c]{
\begin{minipage}{0.6\linewidth}
  \includegraphics[width=\linewidth,natwidth=473,natheight=157]{images/3.jpg}
\end{minipage}}
\paragraph{}

\begin{equation*}
  dF_M = -\upsilon\frac{\mu_0}{2\pi} \frac{IdQ}{x\sqrt{1-\frac{\upsilon^2}{c^2}}}, \quad dQ = \rho_2Sdx_2,
  \quad dF_M = -\upsilon n_2eS\frac{\mu_0}{2\pi} \frac{I_1dx_2}{x\sqrt{1-\frac{\upsilon^2}{c^2}}}
\end{equation*}
\paragraph{}

Получим формулу \textbf{силы магнитного взаимодействия двух параллельных проводов:}

\begin{equation*}
  dF_M = -\frac{\mu_0}{2\pi} \frac{I_1I_2}{x\sqrt{1-\frac{\upsilon^2}{c^2}}} dx
\end{equation*}
\paragraph{}

Без доказательства примем на веру следующие утверждения:

\begin{equation*}
  \oint_SBds = 0, \quad \oint_lBdl = \mu_0I
\end{equation*}

Второе равенство также называется \textbf{теоремой о циркуляции:} Пусть есть замкнутый ток $I$ и взят некий контур.
Определим \textbf{циркуляцию}, как сумму всех $\vec{B}\cdot d\vec{l}$. Тогда циркуляция в этом контуре равна $\mu_0I$.

\paragraph{}

Магнитное поле является не потенциальным, а вихревым. Это значит, что его силовые линии замкнуты,
а циркуляция отлична от нуля на контуре, который охватывает ток.

\paragraph{}

Рассчитаем магнитное поле, создаваемое бесконечным проводом во всём пространстве:
\paragraph{}
\begin{minipage}{0.4\linewidth}
  \includegraphics[width=\linewidth,natwidth=364,natheight=266]{images/4.jpg}
\end{minipage}
\begin{minipage}{0.5\linewidth}
  \begin{equation*}
    B2\pi r = \mu_0I \Rightarrow B = \frac{\mu_0I}{2\pi r}
  \end{equation*}
\end{minipage}

\paragraph{}

Для магнитного поля справедлив принцип суперпозиции: $\vec{B} = \Sigma\vec{B_i}$

\paragraph{Список литературы:}

\begin{itemize}
\item
  Калашников С.Г. Электричество
\item
  Зильберман Г.Е. Электричество и магнетизм
\item
  Сивухин Д.В. Общий курс физики. Т.3. Электричество
\end{itemize}

\subsection{Электромагнитная индукция}
\paragraph{}

В 1831 году Майкл Фарадей провёл следующий опыт:

\paragraph{}
  \noindent\makebox[\linewidth][c]{
  \begin{tikzpicture}
    \draw (0,0)
    to node[draw,circle,fill=white] {A} (0,2)
    to [short] (2,2)
    to [L] (2,0)
    to [short] (0,0);
  \end{tikzpicture}}

Он подключил амперметр к катушке и заметил, что если вводить в неё постоянный магнит, то в цепи появляется ток.
Так было открыто явление \textbf{электромагнитной индукции}.

\paragraph{}

\textbf{Электромагнитная индукция} заключается в том, что переменное магнитное поле
порождает вихревое электрическое. \textbf{Закон Ленца} гласит, что возникающий при этом индукционный ток
направлен в противодействие причинам, его породившим.

\paragraph{}

Определим магнитный поток как произведение магнитной индукции на площадь $S$ и на косинус угла между вектором
$\vec{B}$ и нормалью $n$:

\begin{equation*}
  \Phi = BScos\alpha, \quad
  B = 1 [\text{Тл}], \quad
  \Phi = 1 [\text{Вб}]
\end{equation*}
\begin{equation*}
  \text{ЭДС индукции } \varepsilon_i = -\frac{d\Phi}{dt}
\end{equation*}
\paragraph{}

\noindent\makebox[\linewidth][c]{
\begin{minipage}{0.6\linewidth}
  \textbf{Задача.} Перемычка длины $x$ движется вправо со скоростью $\vec{\upsilon}$.
  \begin{equation*}
    \varepsilon_i = -\frac{\Delta\Phi}{\Delta t} = -\frac{B\Delta S}{\Delta t} =
    -\frac{Bx\upsilon\Delta t}{\Delta t} = -Bx\upsilon
  \end{equation*}
\end{minipage}
\begin{minipage}{0.4\linewidth}
  \includegraphics[width=\linewidth,natwidth=368,natheight=181]{images/5.jpg}
\end{minipage}}
\paragraph{}

Тогда возникший индукционный ток будет равен I = $\frac{Bx\upsilon}{R}$ и направлен (по правилу буравчика)
против часовой стрелки.

\begin{equation*}
  \varepsilon_i = IR; \quad \frac{Q}{\Delta t} = \frac{I^2R\Delta t}{\Delta t} = P = \frac{B^2x^2\upsilon^2}{R}
\end{equation*}

\paragraph{Индуктивность катушки.}

Пусть есть катушка. $N$ --- количество витков, $S$ --- площадь сечения.

\begin{equation*}
  \Phi = BSN = \underbrace{\alpha SN}_{L}I
\end{equation*}

\textbf{Индуктивность} $L$ --- коэффициент пропорциональности между электрическим током $I$, текущим в каком-либо
замкнутом контуре, и магнитным потоком $\Phi$.

\begin{equation*}
  L = 1[\text{Гн}]
\end{equation*}

Для соленоида длины $l$ индуктивность равна $L = \frac{l}{b}$, где $b$ --- расстояние между витками. Если же $b = 0$,
то $L = \frac{l}{d}$, где $d$ --- толщина витка.

\begin{equation*}
  \varepsilon_i = -\Phi' = -LI
\end{equation*}

\newpage

\subsection{Движение в электромагнитных полях}

\begin{equation*}
  \vec{F_{\text{Л}}} = \underbrace{q\vec{E}}_{F_\text{ЛЭ}} + \underbrace{q\left[\vec{\upsilon}\times\vec{B}\right]}_{F_\text{ЛМ}}
\end{equation*}
\paragraph{}

Распишем векторное произведение как определитель матрицы:

\begin{equation*}
  \vec{F}_{\text{ЛМ}}
  =
  q
  \begin{vmatrix}
    \vec{i} & \vec{j} & \vec{k}\\
    \upsilon_x & \upsilon_y & \upsilon_z\\
    B_x & B_y & B_z
  \end{vmatrix}
  =
  q\left(
  (\upsilon_yB_z - \upsilon_zB_y)\vec{i}
  +
  (\upsilon_ZB_x - \upsilon_xB_z)\vec{j}
  +
  (\upsilon_xB_y - \upsilon_yB_x)\vec{k}
  \right)
\end{equation*}
\paragraph{}

Получили полезную формулу магнитной составляющей силы Лоренца.

\paragraph{Задача 1.}

Точечный заряд $q$ движется в некоторой плоскости с начальной скоростью $\upsilon$. Вектор магнитной
индукции перпендикулярен плоскости. Определить характеристики траектории движения частицы.

\paragraph{}

\noindent\makebox[\linewidth][c]{
\begin{minipage}{0.6\linewidth}
  Заметим, что $\upsilon = const$, потому что вектор силы Лоренца перпендикулярен $\vec{\upsilon}$, значит
  траектория движения частицы - окружность.
\end{minipage}
\begin{minipage}{0.4\linewidth}
  \includegraphics[width=\linewidth,natwidth=342,natheight=188]{images/6.jpg}
\end{minipage}}

Запишем второй закон Ньютона:

\begin{equation*}
  \frac{m\upsilon^2}{R} = q\upsilon B \Rightarrow R = \frac{m\upsilon}{qB}
\end{equation*}

Найдём период обращения $T$:

\begin{equation*}
  T = \frac{2\pi R}{\upsilon} = \frac{2\pi m}{qB}.
\end{equation*}
\paragraph{}

Покажем пример применения полученных результатов:

\paragraph{Задача 2.}

Электрон влетает в однородное магнитное поле ширины $R$. Необходимо определить минимальную скорость,
при которой электрон вылетит из поля.

\paragraph{}

\noindent\makebox[\linewidth][c]{
\begin{minipage}{0.5\linewidth}
  Скорость будет минимальна при такой траектории электрона, что на выходе из поля электрон будет лететь по касательной.
  Так как частица будет лететь по окружности, радиус этой окружности будет равен R.
  \begin{equation*}
    \frac{m\upsilon}{eB} = R \Rightarrow \upsilon = \frac{ReB}{m}.
  \end{equation*}
\end{minipage}
\begin{minipage}{0.5\linewidth}
  \includegraphics[width=\linewidth,natwidth=358,natheight=309]{images/7.jpg}
\end{minipage}}

\paragraph{Задача 3.}

Точечный заряд имеет произвольный начальный вектор скорости, а вектор $\vec{B}$ перпендикулярен
некоторой плоскости $\gamma$.
Необходимо понять, как будет двигаться заряд.

\noindent\makebox[\linewidth][c]{
\begin{minipage}{0.5\linewidth}
  Вектор $\vec{\upsilon}$ можно разложить на две составляющие: одна лежит в плоскости $\gamma$,
  другая перпендикулярна ей.
  \begin{equation*}
    \vec{F_{\text{Л}}} = q \left[ (\vec{\upsilon}_{\perp} + \vec{\upsilon}_{\parallel})\times\vec{B} \right]
  \end{equation*}
\end{minipage}
\begin{minipage}{0.5\linewidth}
  \includegraphics[width=\linewidth,natwidth=383,natheight=230]{images/8.jpg}
\end{minipage}}
\paragraph{}

Если перейти в систему отсчёта, движущуюся вверх со скоростью $\vec{\upsilon}_\parallel$, то в ней траектория частицы
будет окружностью, значит, если вернуться в исходную систему отсчёта, то частица будет двигаться по спирали.

\begin{equation*}
  \upsilon_\parallel = \upsilon cos\alpha, \quad \upsilon_\perp = \upsilon sin\alpha,
  \quad R = \frac{m\upsilon sin\alpha}{qB}, \quad T = \frac{2\pi m}{qB}, \quad h = \frac{2\pi m}{qB} \upsilon cos\alpha
\end{equation*}

\paragraph{Задача 4.}

То же условие, что и в задаче 3, но сонаправленно с $\vec{B}$ действует электрическое поле $\vec{E}$.

\paragraph{}

В этой задаче $\vec{\upsilon_\parallel}$ увеличивается линейно.

\begin{equation*}
  a_z = \frac{qE}{m}, \quad \upsilon_z = \frac{qE}{m}t, \quad z(t) = \frac{qE}{2m}t^2
\end{equation*}

\paragraph{}

\newpage

Рассчитаем расстояние между витками спирали, по которой полетит частица:

\begin{equation*}
  h_n = z_n - z_{n-1} = \frac{qE}{2m}\left(\frac{2\pi m}{qB}\right)^2\left(n^2-(n-1)^2\right) = \frac{2\pi^2Em}{qB}(2n-1)
\end{equation*}

\paragraph{Закон Био-Савара-Лапласа} Вектор магнитной индукции можно определить по формуле:

\begin{equation*}
  B = \int \frac{\mu_0}{4\pi} \frac{ I[\vec{r} \times d \vec{l}] }{r^3} =
  \frac{\mu_0I}{4\pi}\int\frac{[\vec{r}\times d\vec{l}]}{r^3},
\end{equation*}
где $\mu_0 = 4\pi \times 10^{-7}$ Гн/м

\newpage
\noindent\makebox[\linewidth]{\rule{\paperwidth}{0.4pt}}
\section{(15.09.18)}
\noindent\makebox[\linewidth]{\rule{\paperwidth}{0.4pt}}

\subsection{Переменный ток}
\paragraph{}

Переменный ток --- это ток, сила и направление которого меняется со временем. Мы будем рассматривать
ток, меняющийся по гармоническому закону

\begin{equation*}
  i = i_0sin(\omega t)
\end{equation*}

\noindent\makebox[\linewidth][c]{\rule{\linewidth}{0.4pt}}
\paragraph{}
\noindent\makebox[\linewidth][c]{
  \begin{minipage}{0.5\linewidth}
    \noindent\makebox[\linewidth][c]{
    \begin{circuitikz}[circuit ee IEC]
      \draw (2,0) to [ac source] (0,0) to [short={near end}] (0,1)
      to [resistor={info=$R$}] (2,1) to [short={near end}] (2,0);
    \end{circuitikz}}
\end{minipage}
\begin{minipage}{0.5\linewidth}
  \begin{equation*}
    U = iR \enskip \Rightarrow \enskip U = i_0Rsin(\omega t)
  \end{equation*}
\end{minipage}}

\paragraph{}
\noindent\makebox[\linewidth][c]{\rule{\linewidth}{0.4pt}}
\paragraph{}

\noindent\makebox[\linewidth][c]{
  \begin{minipage}{0.5\linewidth}
    \noindent\makebox[\linewidth][c]{
    \begin{circuitikz}[circuit ee IEC]
      \draw (2,0) to [ac source] (0,0) to [short={near end}] (0,1)
      to [capacitor={info=$C$}] (2,1) to [short={near end}] (2,0);
    \end{circuitikz}}
\end{minipage}
\begin{minipage}{0.5\linewidth}
  \begin{equation*}
    U = \frac{q}{c} = \frac{1}{c}\int i_osin(\omega t)dt = -\frac{1}{\omega c}cos(\omega t)
  \end{equation*}
  \paragraph{}
  Введём ёмкостное сопротивление:
  \begin{equation*}
    X_c = \frac{1}{\omega c}
  \end{equation*}
\end{minipage}}

\paragraph{}
\noindent\makebox[\linewidth][c]{\rule{\linewidth}{0.4pt}}
\paragraph{}

\noindent\makebox[\linewidth][c]{
  \begin{minipage}{0.5\linewidth}
    \noindent\makebox[\linewidth][c]{
    \begin{circuitikz}[circuit ee IEC]
      \draw (2,0) to [ac source] (0,0) to [short={near end}] (0,1)
      to [inductor={info=$L$}] (2,1) to [short={near end}] (2,0);
    \end{circuitikz}}
\end{minipage}
  \begin{minipage}{0.5\linewidth}
  \begin{equation*}
    \varepsilon = Li' = L\omega i_0cos(\omega t)
  \end{equation*}
  Введём индуктивное сопротивление:
  \begin{equation*}
    X_L = \omega L
  \end{equation*}
\end{minipage}}

\paragraph{}
\noindent\makebox[\linewidth][c]{\rule{\linewidth}{0.4pt}}
\paragraph{}

В цепях переменного тока выделяют \textbf{активное} и \textbf{реактивное сопротивление}.
На активном сопротивлении выделяется джоулево тепло.
Реактивное сопротивление обусловлено передачей энергии переменным током электрическому или магнитному полю.

\paragraph{}

Реактивное сопротивление элемента обозначается как $X$ и равно разности индуктивного и ёмкостного сопротивления
элемента.

\begin{equation*}
  X = X_L - X_C
\end{equation*}

В зависимости от знака $X$ можно судить о свойствах элемента в цепи:

\begin{itemize}
\item
  $X > 0$ --- элемент проявляет свойства индуктивности.
\item
  $X = 0$ --- элемент имеет чисто активное сопротивление.
\item
  $X < 0$ --- элемент проявляет ёмкостные свойства.
\end{itemize}

\paragraph{}
\textbf{Электрический импеданс} --- это комплексное число $Z = R + jX$, где $R$ --- активное сопротивление, $j$ ---
мнимая единица (чтобы не путать с обозначением мгновенной силы тока $i$). Аргумент импеданса есть не что иное, как
фазовый сдвиг между током и напряжением электрической цепи.

\paragraph{}

Модуль импеданса по теореме Пифагора:

\begin{equation*}
  |Z| = \sqrt{R^2 + (\omega L - \frac{1}{\omega C})^2}
\end{equation*}

\paragraph{}

Зачем нужен импеданс: с помощью него можно применять закон Ома для участка цепи с переменным током. Например, если
известна амплитуда $U$ напряжения в цепи и полное сопротивление (импеданс) $Z$ её элементов, то амплитуда силы тока в
цепи будет вычисляться по формуле

\begin{equation*}
  I = \frac{U}{Z}
\end{equation*}

\paragraph{}

Рассмотрим подробнее поведение конденсатора в цепи с переменным током. Заряд на обкладках является
максимальным на экстремумах синусоиды. Между экстремумами конденсатор сначала разряжается, а потом заряжается.
В итоге за один период он два раза заряжается и два раза разряжается. Поэтому говорят, что конденсатор проводит
переменный ток (хотя на самом деле никакой заряд через него не проходит, но ток в цепи существует, как если бы в ней
не было разрыва).

\subsection{Колебательный контур}
\paragraph{}

Пусть к заряженному конденсатору присоединена катушка индуктивности.


\noindent\makebox[\linewidth][c]{
  \begin{minipage}{0.5\linewidth}
    \noindent\makebox[\linewidth][c]{
    \begin{circuitikz}[circuit ee IEC]
      \draw (2,0) to [capacitor={info=$C$}] (0,0) to [short={near end}] (0,1)
      to [inductor={info=$L$}] (2,1) to [short={near end}] (2,0);
    \end{circuitikz}}
\end{minipage}
\begin{minipage}{0.5\linewidth}
  \begin{equation*}
    \frac{q}{c} = -Li', \quad \ddot{q} + \frac{q}{LC} = 0
  \end{equation*}
  \begin{equation*}
    \omega_0 = \frac{1}{LC}, \quad T = 2\pi\sqrt{LC}
  \end{equation*}
\end{minipage}}
\paragraph{}

Эта схема представляет собой \textbf{колебательный контур} и часто используется в радиоэлектронике.

\paragraph{}

Рассчитаем энергию магнитного поля $W$, порождаемого колебательным контуром, когда ток изменяется с $i$ до 0.
ЭДС самоиндукции, возникающая в катушке, равна

\begin{equation*}
  \varepsilon = -\Phi = -Li' = -L\frac{\Delta i}{\Delta t}
\end{equation*}
\paragraph{}

Величина тока в цепи равна

\begin{equation*}
  i = \frac{\varepsilon}{R} = -\frac{L}{R}\frac{\Delta i}{\Delta t}
\end{equation*}
\paragraph{}

За промежуток времени $\Delta t$ в цепи выделится энергия $\Delta Q$

\begin{equation*}
  \Delta Q = i^2R\Delta t = -Li\Delta i = -\Phi\Delta i
\end{equation*}
\paragraph{}

Вся выделившаяся энергия получается интегрированием

\begin{equation*}
  Q = \int_{i}^{0}\Delta Q = \frac{Li^2}{2} = W
\end{equation*}

\subsection{Магнитное поле в веществе}
\paragraph{}

Для некоторого плоского контура введём величину $\mu = iS$, которая называется \textbf{магнитным моментом}.
Вектор магнитного момента перпендикулярен плоскости контура. Направление выбирается правилом буравчика
(буравчик необходимо закручивать по движению тока).

\paragraph{}

Электрон, вращающийся вокруг ядра атома, создаёт элементарный круговой ток и магнитное поле. У этой системы есть
некоторый магнитный момент $\mu$. Тогда \textbf{намагниченность} системы по определению равна магнитному моменту
единицы объема вещества:
\begin{equation*}
  M = \frac{\mu}{V}
\end{equation*}

\paragraph{}

Среди веществ по магнитным свойствам выделяют три типа:

\begin{enumerate}
\item
  Диамагнетики --- вещества, которые намагничиваются \textit{против направления} внешнего магнитного поля.
\item
  Парамагнетики --- вещества, которые намагничиваются \textit{в направлении} внешнего магнитного поля.
\item
  Ферромагнетики --- вещества, которые способны иметь остаточную намагниченность в отсутствие внешнего магнитного поля.
\end{enumerate}
\paragraph{}

Рассмотрим отношение внутреннего и внешнего магнитного поля разных типов веществ:
\paragraph{}
\noindent\makebox[\linewidth][c]{
  \begin{minipage}{0.32\linewidth}
  \noindent\makebox[\linewidth][c]{\textbf{Диа- и парамагнетики}}
  \begin{figure}[H]
    \includegraphics[width=\linewidth,natwidth=246,natheight=264]{images/9.jpg}
  \end{figure}
\end{minipage}
\hspace{1em}
\rulesep
\begin{minipage}{0.42\linewidth}
  \noindent\makebox[\linewidth][c]{\textbf{Ферромагнетики}}
  \begin{figure}[H]
    \includegraphics[width=\linewidth,natwidth=275,natheight=224]{images/10.jpg}
  \end{figure}
\end{minipage}}
\paragraph{}

При увеличении внешнего магнитного поля внутреннее поля ферромагнетика насыщается до некоторого
предела $B_{\text{насыщ}}$.
После постепенного снижения внешнего поля до нуля внутреннее поле ферромагнетика останется больше нуля.
Сила, которую надо приложить, чтобы размагнитить ферромагнетик, называется \textbf{коэрцетивной силой}.

\paragraph{}

Получаем следующий график, который называется петлёй гистерезиса:

\paragraph{}
\noindent\makebox[\linewidth][c]{
\begin{minipage}{0.7\linewidth}
  \includegraphics[width=\linewidth,natwidth=584,natheight=541]{images/11.jpg}
\end{minipage}}
\paragraph{}

Группы атомов, у которых одинаковое направление магнитного момента, называются \textbf{доменами}.

\noindent\makebox[\linewidth][c]{
\begin{minipage}{0.4\linewidth}
  \includegraphics[width=\linewidth,natwidth=299,natheight=169]{images/12.jpg}
\end{minipage}}
\paragraph{}

У ненамагниченного вещества сумма магнитных моментов равна нулю. При намагничивании вещества,
магнитные моменты доменов становятся направленными в одну сторону.

\paragraph{}

\textbf{Температура Кюри} --- температура, при которой ферромагнетик теряет свойство остаточной намагниченности.

\paragraph{}

Если рассматривать магнитные моменты соседних частиц вещества, то можно выделить три группы:

\begin{itemize}
\item
  Ферромагнетики --- б\'{о}льшая часть магнитных моментов сонаправлена. ($\uparrow \uparrow \uparrow \uparrow$)
\item
  Антиферромагнетики --- магнитные моменты антипараллельны друг другу и примерно равны по модулю,
  суммарная намагниченность очень мала. ($\uparrow \downarrow \uparrow \downarrow$)
\item
  Ферримагнетики --- магнитные моменты антипараллельны друг другу, но не равны по модулю.
  ($\uparrow_\downarrow\uparrow_\downarrow$)
\end{itemize}

\end{document}
