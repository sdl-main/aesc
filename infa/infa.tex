\documentclass{article}
\author{SDL}

\usepackage{subcaption}
\usepackage{listings}
\usepackage{amsmath}
\usepackage{epigraph}
\usepackage{titlesec}
\usepackage{color}
\usepackage[T1,T2A]{fontenc}
\usepackage[utf8]{inputenc}
\usepackage[english,russian]{babel}

\definecolor{mygreen}{rgb}{0,0.6,0}
\definecolor{mygray}{rgb}{0.5,0.5,0.5}
\definecolor{mymauve}{rgb}{0.58,0,0.82}

\lstset{ 
  backgroundcolor=\color{white},   % choose the background color; you must add \usepackage{color} or \usepackage{xcolor}; should come as last argument
  basicstyle=\ttfamily,        % the size of the fonts that are used for the code
  breakatwhitespace=false,         % sets if automatic breaks should only happen at whitespace
  breaklines=true,                 % sets automatic line breaking
  captionpos=b,                    % sets the caption-position to bottom
  commentstyle=\color{mygreen},    % comment style
  deletekeywords={...},            % if you want to delete keywords from the given language
%  escapeinside={\%*}{*)},          % if you want to add LaTeX within your code
  extendedchars=true,              % lets you use non-ASCII characters; for 8-bits encodings only, does not work with UTF-8
  frame=single,	                   % adds a frame around the code
  keepspaces=true,                 % keeps spaces in text, useful for keeping indentation of code (possibly needs columns=flexible)
  keywordstyle=\color{blue},       % keyword style
  language=C++,                 % the language of the code
  morekeywords={*,...},            % if you want to add more keywords to the set
  numbers=left,                    % where to put the line-numbers; possible values are (none, left, right)
  numbersep=5pt,                   % how far the line-numbers are from the code
  numberstyle=\tiny\color{mygray}, % the style that is used for the line-numbers
  rulecolor=\color{black},         % if not set, the frame-color may be changed on line-breaks within not-black text (e.g. comments (green here))
  showspaces=false,                % show spaces everywhere adding particular underscores; it overrides 'showstringspaces'
  showstringspaces=false,          % underline spaces within strings only
  showtabs=false,                  % show tabs within strings adding particular underscores
  stepnumber=1,                    % the step between two line-numbers. If it's 1, each line will be numbered
  stringstyle=\color{mymauve},     % string literal style
  tabsize=4,	                   % sets default tabsize to 2 spaces
  title=\lstname                   % show the filename of files included with \lstinputlisting; also try caption instead of title
}

\titleformat{\section}
  {\normalfont\Large\bfseries}   % The style of the section title
  {}                             % a prefix
  {0pt}                          % How much space exists between the prefix and the title
  {Лекция \thesection\quad}    % How the section is represented



\begin{document}

\noindent\makebox[\linewidth]{\rule{\paperwidth}{0.4pt}}
\section{(04.09.2018)}
\noindent\makebox[\linewidth]{\rule{\paperwidth}{0.4pt}}

\subsection{Рекурсия}

\epigraph{Чтобы понять рекурсию, надо понять рекурсию.}

\paragraph{}
Рекурсивный алгоритм для вычисления вызывает себя.
\paragraph{}

Виды рекурсии:

\begin{enumerate}
\item
  Прямая рекурсия - функция A вызывает сама себя.
\item
  Косвенная рекурсия - А вызывает B, A вызывает B.
  В одной функции обязательно должна быть хотя бы одна нерекурсивная ветка
\end{enumerate}

\subsection{Процесс рекурсивного вычисления}
\paragraph{}

Этапы процесса рекурсивного вычисления:

\begin{enumerate}
\item
  Погружение в рекурсию.
\item
  Всплывание из рекурсии.
\end{enumerate}

\subsection{Суперкосинус}
\begin{equation*}
  supercos(x, n) = \underbrace{cos...cos}_n(x);
\end{equation*}

\begin{lstlisting}
  double supercos(double x, int n) {
      if (n == 0)
          return x;
      return cos(supercos(x, n-1));
  }
\end{lstlisting}

\subsection{Алгоритм Евклида}
\paragraph{}

Находит наибольший общий делитель чисел a и b.

\begin{lstlisting}
  int gcd(int a, int b) {
      if (b == 0) return a;
      return gcd(b, a % b);
  }
\end{lstlisting}

\subsection{Функция Аккермана}

Работает на множестве натуральных чисел.

\begin{equation*}
  A(m, n) =
  \begin{cases}
    n + 1, & m = 0\\
    A(m - 1, 1), & m > 0, n = 0\\
    A(m - 1, A(m, n - 1)), & m > 0, n > 0
  \end{cases}  
\end{equation*}

\subsection{Разбор строки по грамматике}
\paragraph{Нотационные формы Бэкуса-Наура}
\begin{itemize}
\item
  <цифра> ::= 0|1|...|9
\item
  <буква> ::= a|...|z|A|...|Z|\_
\item
  <идентификатор> ::= <буква> | <идентификатор><буква> | <идентификатор><цифра>
\item
  <оператор> ::= <выражение> | <if> | <while> | ...
\item
  <while> ::= while( <выражение> ) <оператор>
\item
  <формула> ::= <цифра> | (<формула> <знак> <формула>)
\item
  <знак> ::= +|-|*
\end{itemize}

\paragraph{Задачи:}

\begin{enumerate}
\item
  Вычислить значение корректного выражения
\item
  Проверить корректность выражения.
\end{enumerate}

\newpage
Решаем первую задачу:

\begin{lstlisting}
  int form() {
      char c, z;
      int x, y;
      cin >> c;
      if (c >= '0' && c <= '9') return c - '0';
      x = form();
      cin >> z;
      y = form();
      cin >> c;
      if (z == '+') return x + y;
      if (z == '-') return x - y;
      if (z == '*') return x * y;
  }
\end{lstlisting}

\paragraph{Общие принципы вычисления выражений:}

\begin{enumerate}
\item
  Для каждого понятия есть своя функция.
\item
  Если понятие нетривиальное - вызываем его функцию.
\item
  Тривиальные понятия анализируем явно.
\end{enumerate}



\end{document}
