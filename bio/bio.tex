\documentclass[dvipdfmx]{article}

\usepackage{amsmath}
\usepackage{subcaption}
\usepackage{graphicx}
\usepackage{titlesec}
\usepackage{tikz}
\usepackage{caption}
\usepackage{longtable}
\usepackage{float}
\usepackage{circuitikz}
\usepackage[T1,T2A]{fontenc}
\usepackage[utf8]{inputenc}
\usepackage[english,russian]{babel}

\titleformat{\section}
            {\normalfont\Large\bfseries}
            {}
            {0pt}
            {Урок \thesection\quad}

\begin{document}

\noindent\makebox[\linewidth]{\rule{\paperwidth}{0.4pt}}
\section{(19.09.18)}
\noindent\makebox[\linewidth]{\rule{\paperwidth}{0.4pt}}

\subsection{Расщепление признаков в потомстве при гибридизации.}
\paragraph{}

Наследуются не признаки, а гены. \textbf{Ген} - участок ДНК, в котором записан один белок.

\begin{itemize}
\item
  Первичный признак организма - белок.
\item
  Вторичный признак - всё, что угодно (например, цвет).
\end{itemize}

\paragraph{}
Ген $\rightarrow$ белок $\rightarrow$ (химические реакции) $\rightarrow$ признак

\paragraph{}
Варианты генов называются \textit{аллелями}.

\paragraph{}
Мы и горох - диплоидные организмы (двойной набор генов).

\paragraph{}
А - \textit{доминантный аллель}, а - \textit{рецессивный}.
Рецессивный признак проявляется только при отсутствии доминантного.

\paragraph{}
Например, АА, Аа, аА - желтый горох, аа - зеленый

\paragraph{}
\textit{Гомозиготные} особи - АА, аа, 
\textit{Гетерозиготные} особи - Аа

\paragraph{}
\textbf{Генотип} - совокупность всех генов особи
\paragraph{}
\textbf{Фенотип} - совокупность всех признаков особи


\begin{longtable}[c]{|p{1cm}|p{1cm}|p{2cm}|}
  \caption{Группы крови}\\
  \hline
  I   &  0  &  $i^0i^0$\\
  \hline
  II  &  A  &  $I^AI^A, I^Ai^0$\\
  \hline
  III &  B  &  $I^BI^B, I^Bi^0$\\
  \hline
  IV  &  AB &  $I^AI^B$\\
  \hline
\end{longtable}

\newpage
\noindent\makebox[\linewidth]{\rule{\paperwidth}{0.4pt}}
\section{(26.09.18)}
\noindent\makebox[\linewidth]{\rule{\paperwidth}{0.4pt}}

\subsection{Дигибридное скрещивание и взаимодействие генов}

\paragraph{}
Человек с пищей получает вещество А. Ген А превращает его в вещество В. Ген В превращает это вещество в эумеланин (черный пигмент), а ген С превращает это вещество в феомеланин (рыжий пигмент). Если не работает ген А --- человек
альбинос, если не работает ген В --- человек рыжий, если не работает ген С --- человек чернокожий.

\paragraph{}
Рецессивные гены кодируют неработающие ферменты.

\paragraph{}
Разные аллели --- результат мутаций в гене.

\paragraph{}
У одного гена может быть не два разных аллеля, а целое множество.
Например, дикая окраска кроликов $C_1$, гималайская $C_2$, альбинизм $C_3$. Варианты серой раскраски: $C_1C_1$, $C_1C_2$, $C_1C_3$.
Варианты белой раскраски: $C_3C_3$. Варианты гималайской раскраски: $C_2C_2$, $C_2C_3$.

\paragraph{}
Расщепление при дигибридном скрещивании (особи отличаются по двум двуаллельным генам) равно 9:3:3:1


\end{document}
