\documentclass[dvipdfmx]{article}

\usepackage{amsmath}
\usepackage{subcaption}
\usepackage{graphicx}
\usepackage{titlesec}
\usepackage{tikz}
\usepackage{caption}
\usepackage{longtable}
\usepackage{float}
\usepackage{circuitikz}
\usepackage[T1,T2A]{fontenc}
\usepackage[utf8]{inputenc}
\usepackage[english,russian]{babel}

\titleformat{\section}
            {\normalfont\Large\bfseries}
            {}
            {0pt}
            {Урок \thesection\quad}

\begin{document}

\noindent\makebox[\linewidth]{\rule{\paperwidth}{0.4pt}}
\section{(19.09.18)}
\noindent\makebox[\linewidth]{\rule{\paperwidth}{0.4pt}}

\subsection{Расщепление признаков в потомстве при гибридизации.}
\paragraph{}

Наследуются не признаки, а гены. \textbf{Ген} - участок ДНК, в котором записан один белок.

\begin{itemize}
\item
  Первичный признак организма - белок.
\item
  Вторичный признак - всё, что угодно (например, цвет).
\end{itemize}

\paragraph{}
Ген $\rightarrow$ белок $\rightarrow$ (химические реакции) $\rightarrow$ признак

\paragraph{}
Варианты генов называются \textit{аллелями}.

\paragraph{}
Мы и горох - диплоидные организмы (двойной набор генов).

\paragraph{}
А - \textit{доминантный аллель}, а - \textit{рецессивный}.
Рецессивный признак проявляется только при отсутствии доминантного.

\paragraph{}
Например, АА, Аа, аА - желтый горох, аа - зеленый

\paragraph{}
\textit{Гомозиготные} особи - АА, аа, 
\textit{Гетерозиготные} особи - Аа

\paragraph{}
\textbf{Генотип} - совокупность всех генов особи
\paragraph{}
\textbf{Фенотип} - совокупность всех признаков особи


\begin{longtable}[c]{|p{1cm}|p{1cm}|p{2cm}|}
  \caption{Группы крови}\\
  \hline
  I   &  0  &  $i^0i^0$\\
  \hline
  II  &  A  &  $I^AI^A, I^Ai^0$\\
  \hline
  III &  B  &  $I^BI^B, I^Bi^0$\\
  \hline
  IV  &  AB &  $I^AI^B$\\
  \hline
\end{longtable}

\newpage
\noindent\makebox[\linewidth]{\rule{\paperwidth}{0.4pt}}
\section{(26.09.18)}
\noindent\makebox[\linewidth]{\rule{\paperwidth}{0.4pt}}

\subsection{Дигибридное скрещивание и взаимодействие генов}

\paragraph{}
Человек с пищей получает вещество А. Ген А превращает его в вещество В. Ген В превращает это вещество в эумеланин (черный пигмент), а ген С превращает это вещество в феомеланин (рыжий пигмент). Если не работает ген А --- человек
альбинос, если не работает ген В --- человек рыжий, если не работает ген С --- человек чернокожий.

\paragraph{}
Рецессивные гены кодируют неработающие ферменты.

\paragraph{}
Разные аллели --- результат мутаций в гене.

\paragraph{}
У одного гена может быть не два разных аллеля, а целое множество.
Например, дикая окраска кроликов $C_1$, гималайская $C_2$, альбинизм $C_3$. Варианты серой раскраски: $C_1C_1$, $C_1C_2$, $C_1C_3$.
Варианты белой раскраски: $C_3C_3$. Варианты гималайской раскраски: $C_2C_2$, $C_2C_3$.

\paragraph{}
Расщепление при дигибридном скрещивании (особи отличаются по двум двуаллельным генам) равно 9:3:3:1

\newpage
\noindent\makebox[\linewidth]{\rule{\paperwidth}{0.4pt}}
\section{(10.10.18)}
\noindent\makebox[\linewidth]{\rule{\paperwidth}{0.4pt}}

\subsection{Регуляция работы генов}

\paragraph{Основы молекулярной биологии}
\paragraph{}

ДНК, РНК и белки - полимеры, состоящие из мономеров.
Мономеры ДНК и РНК --- \textbf{нуклеотиды}.

\paragraph{}

У ДНК азотистые основания А, Г, Ц, Т. У РНК вместо Т У.

\paragraph{}

Мономеры белка --- \textbf{аминокислоты}.

\begin{itemize}
\item
  ДНК --- цепочка из нуклеотидов.
\item
  Белок --- цепочка из аминокислот.
\end{itemize}

\paragraph{}
В каждой клетке никогда не работают все гены, поэтому клетки разные.

\paragraph{}
Ген не будет работать, если на \textbf{промотор} не сядет белок, катализирующий считывание информации с
ДНК на РНК (транскрипцию). На конце гена есть \textbf{терминатор}.

\paragraph{}
Механизм регуляции работы генов был вперые выяснен у бактерии кишечной палочки.

\paragraph{}

У бактерий гены объединены в \textit{опероны}.
Оперон --- участок ДНК, транскрипция которого осуществляется на одну матричную РНК.

Оперон содержит структурные гены и регуляторные элементы.
Структурные гены кодируют белки.

\paragraph{}
Регуляторные белки:
\begin{itemize}
\item
  Промотор --- необходим для связывания РНК-полимеразы.
\item
  Терминатор --- необходим для отвязывания РНК-полимеразы.
\item
  Оператор --- на него может сесть белок-регулятор.
\end{itemize}
\paragraph{}
\textbf{Лактоза} --- молочный сахар.

\paragraph{}
Типы регуляторных белков:
\begin{itemize}
\item
  Репрессоры
\item
  Активизаторы
\end{itemize}

Активный белок-репрессор присоединяется к оператору.
Белок-репрессор можно включать и выключать с помощью определенного вещества --- лактозы.

\paragraph{Зачем нужен лактозный оперон}

В лактозном опероне закодированы ферменты, позволяющие бактерии питаться лактозой.

\newpage
\noindent\makebox[\linewidth]{\rule{\paperwidth}{0.4pt}}
\section{(17.10.18)}
\noindent\makebox[\linewidth]{\rule{\paperwidth}{0.4pt}}

\subsection{Механизм регуляции работы генов}

\paragraph{}
Лактозный оперон регулируется белком-репрессором, который присоединяется к оператору и не даёт ферменту, копирующему
информацию с ДНК и РНК начать копировать информацию.

\paragraph{}
В лактозном опероне закодированы ферменты, позволяющие бактерии питаться лактозой. Когда в среде нет лактозы, бактерия
может выключить транскрипцию лактозного оперона. При отсутствии в среде лактозы белок-репрессор активен. Когда лактоза
есть, она связывается с белком-репрессором и деактивирует его.

\paragraph{}
Наши клетки происходят от архей, а митохондрии происходят от бактерий.

\paragraph{}
Лактоза расщепляется на молекулы глюкозы.
В лактозном опероне есть оператор, с которым связывается \textbf{белок-активатор катаболизма (БАК)}.
Фермент РНК-полимераза может транскрибировать оперон только если БАК связан со своим оператором.
Когда глюкоза есть, БАК неактивен.

\begin{itemize}
\item
  Лактозы нет, глюкоза есть: БАК неактивен, репрессор активен.
\item
  Лактоза есть, глюкоза есть: БАК неактивен, репрессор неактивен.
\item
  Лактоза есть, глюкозы нет: БАК активен, репрессор неактивен.
\item
  Лактозы нет, глюкозы нет: БАК активен, репрессор активен.
\end{itemize}

\subsection{Наследование пола}

У многих животных есть \textbf{половые хромосомы} --- одна пара у большинства диплоидных организмов.
Все остальные хромосомы называются \textbf{аутосомы}.
У человека половые хромосомы имеют номер 23.

\paragraph{}
У человека и других млекопитающих:
\begin{itemize}
\item
  $XX$ --- женский пол (гомогаметный)
\item
  $XY$ --- мужской пол (гетерогаметный)
\end{itemize}

\subsection{Признаки, сцепленные с полом}

\paragraph{}
Дальтонизм --- сцепленный с $X$-хромосомой рецессивный признак. У женщины проявится дальтонизм, только если рецессивный
аллель встретится на обеих $X$-хромосомах. Мужчина будет дальтоником, если в $X$-хромосоме есть рецессивный аллель.

\paragraph{}
Гемофилия также является признаком, сцепленным с полом. 

\newpage
\noindent\makebox[\linewidth]{\rule{\paperwidth}{0.4pt}}
\section{(24.10.18)}
\noindent\makebox[\linewidth]{\rule{\paperwidth}{0.4pt}}

\subsection{Сцепленное наследование}

\paragraph{}
Наследование генов, которые находятся на одной хромосоме недалеко друг от друга.

\paragraph{}
Гены, находящиеся в одной хромосоме недалеко друг от друга, наследуются сцепленно, но иногда сцепленность может быть
нарушена в результате \emph{кроссинговера}.

\paragraph{}
\emph{Кроссинговер} --- взаимный обмен участками генов между двумя хромосомами.

\subsection{Изменчивость и селекция}

\paragraph{}
\emph{Селекция} --- прикладная область генетики, связанная с выведением определенных сортов растений и видов животных.
\textbf{ГМО не опасны.}

% МЕЙОЗ

\newpage
\noindent\makebox[\linewidth]{\rule{\paperwidth}{0.4pt}}
\section{(28.11.18)}
\noindent\makebox[\linewidth]{\rule{\paperwidth}{0.4pt}}

\emph{Биологическая эволюция} --- это история развития живой природы.

Жан-Батист Ламарк --- автор одной из первых теорий эволюции (приобретенные при жизни признаки наследуются)

\emph{Естественный отбор} --- это процесс, в ходе которого выживают и размножаются более приспособленные.

Дарвин и Уоллес считали, что основной механизм эволюции --- ествественный отбор.

Современная синтетическая теория эволюции --- синтез теории Дарвина-Уоллеса и молекулярной генетики.

Необходимые условия естественного отбора:
\begin{itemize}
\item
  Наследственность
\item
  Наследственная изменчивость
\item
  Борьба за существование
\end{itemize}

\end{document}
