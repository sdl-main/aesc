\documentclass{article}
\author{SDL}

\usepackage{subcaption}
\usepackage{titlesec}
\usepackage[T1,T2A]{fontenc}
\usepackage[utf8]{inputenc}
\usepackage[english,russian]{babel}

\titleformat{\section}
            {\normalfont\Large\bfseries}
            {}
            {0pt}
            {Урок \thesection\quad}

\begin{document}

\noindent\makebox[\linewidth]{\rule{\paperwidth}{0.4pt}}
\section{(04.09.2018)}
\noindent\makebox[\linewidth]{\rule{\paperwidth}{0.4pt}}


\subsection{Темы итогового сочинения}
\subsubsection{Отцы и дети}
Отцы и дети, Война и мир, Капитанская дочка, Горе от ума, Недоросль, Гроза, Обломов
\subsubsection{Мечта и реальность}
Белые ночи, Преступление и наказание, Алые паруса
\subsubsection{Месть и великодушие}
Война и мир - взаимоотношения Пьера и Долохова, Капитанская дочка, Выстрел (Пушкин), Мастер и Маргарита
\subsubsection{Искусство и ремесло}
Левша, Тупейный художник, Очарованный странник, Мастер и Маргарита, Портрет (Гоголь),
Разговор книгопродавца с поэтом, Лирическое отступление из Мертвых душ,
Блажен незлобливый поэт, Творчество (Ахматова)
\subsubsection{Доброта и жестокость}
\paragraph{Во взаимоотношениях между людьми}
Капитанская дочка, Преступление и наказание, Война и мир, Чучело
\paragraph{В отношениях с животными}
Зачем я убил коростеля, Гуси в полынье, Конь с розовой гривой

\subsection{Критерии оценивания итогового сочинения}
\begin{enumerate}
\item
  Соответствие темы
\item
  Наличие литературных аргументов
\item
  Наличие речевых ошибок
\item
  Наличие грамматических ошибок
\item
  Орфография и пунктуация
\end{enumerate}
Объем сочинения - 350 слов.
Зачет идёт при соотсветствии хотя бы трём критериям.
\subsection{Лев Николаевич Толстой (1828-1910)}

По происхождению аристократ. Со стороны матери родственники были Волконские и Трубецкие.
По линии отца Горчаковы и Толстые. Дед писателя Волконский был военным, очень знатным человеком, своенравным.
Женился на Надежде Сергеевне Трубецкой, у них родилась дочь Мария Николаевна Волконская, мать Толстого.

\paragraph{}

В семье было пятеро детей, Л.Н. был четвертым. Мать умерла при родах пятого ребенка, когда Толстому было два года.
За воспитание детей отвечала дальняя родственница Татьяна Ергольская. 
Прадед Толстого был сподвижником Петра Первого, попал в опалу при Екатерине и был сослан в монастырь.
Дед - Илья Андреич Толстой. Жил в Москве, был членом английского клуба, хлебосольный, открытый, добродушный.
Отец исправил положение дел в семье, женившись на Марии Волконской.

\paragraph{}

Толстой учился в Казанском университете на отделении восточных языков, не сдал историю, перешел на юрфак, но
всё равно не окончил обучение. Уехал в Ясную поляну, начинает заниматься сельским хозяйством, стал помещиком.
Ничего не росло. С этого времени начинает вести дневник.

\paragraph{}

В 1850 году всё бросил и уехал на Кавказ к старшему брату Николаю. Пробыл там около полугода.
Впечатления от этой поездки лягут в основу повести ``Казаки''.
В 1852 году публикует своё первое произведение - повесть ``Детство''.

\paragraph{}

В 1854 году с Кавказа переводится в дунайские войска. Начинается крымская война. Толстой, будучи артиллеристом,
участвует в обороне Севастополя. Впоследствии об этих событиях он пишет рассказы.

\paragraph{}

В 1856 году Толстой выходит в отставку. Останавливается в Петербурге, знакомится с Тургеневым, но не ладит с ним
из-за разгульного образа жизни Толстого. Л.Н. решил вернуться в Ясную поляну с женой.
Его выбор пал на Екатерину, одну из дочерей Тютчева, его очень дальнего родственница. Та его отвергает.

%HW: биографическая таблица Толстого, Война и мир

\end{document}

