\documentclass{article}
\author{SDL}

\usepackage{subcaption}
\usepackage{titlesec}
\usepackage{longtable}
\usepackage{booktabs}
\usepackage[T1,T2A]{fontenc}
\usepackage[utf8]{inputenc}
\usepackage[english,russian]{babel}

\titleformat{\section}
            {\normalfont\Large\bfseries}
            {}
            {0pt}
            {Урок \thesection\quad}

\begin{document}

\noindent\makebox[\linewidth]{\rule{\paperwidth}{0.4pt}}
\section{(04.09.2018)}
\noindent\makebox[\linewidth]{\rule{\paperwidth}{0.4pt}}


\subsection{Темы итогового сочинения}
\subsubsection{Отцы и дети}
Отцы и дети, Война и мир, Капитанская дочка, Горе от ума, Недоросль, Гроза, Обломов
\subsubsection{Мечта и реальность}
Белые ночи, Преступление и наказание, Алые паруса
\subsubsection{Месть и великодушие}
Война и мир - взаимоотношения Пьера и Долохова, Капитанская дочка, Выстрел (Пушкин), Мастер и Маргарита
\subsubsection{Искусство и ремесло}
Левша, Тупейный художник, Очарованный странник, Мастер и Маргарита, Портрет (Гоголь),
Разговор книгопродавца с поэтом, Лирическое отступление из Мертвых душ,
Блажен незлобливый поэт, Творчество (Ахматова)
\subsubsection{Доброта и жестокость}
\paragraph{Во взаимоотношениях между людьми}
Капитанская дочка, Преступление и наказание, Война и мир, Чучело
\paragraph{В отношениях с животными}
Зачем я убил коростеля, Гуси в полынье, Конь с розовой гривой

\subsection{Критерии оценивания итогового сочинения}
\begin{enumerate}
\item
  Соответствие темы
\item
  Наличие литературных аргументов
\item
  Наличие речевых ошибок
\item
  Наличие грамматических ошибок
\item
  Орфография и пунктуация
\end{enumerate}
Объем сочинения - 350 слов.
Зачет идёт при соотсветствии хотя бы трём критериям.
\subsection{Лев Николаевич Толстой (1828-1910)}

По происхождению аристократ. Со стороны матери родственники были Волконские и Трубецкие.
По линии отца Горчаковы и Толстые. Дед писателя Волконский был военным, очень знатным человеком, своенравным.
Женился на Надежде Сергеевне Трубецкой, у них родилась дочь Мария Николаевна Волконская, мать Толстого.

\paragraph{}

В семье было пятеро детей, Л.Н. был четвертым. Мать умерла при родах пятого ребенка, когда Толстому было два года.
За воспитание детей отвечала дальняя родственница Татьяна Ергольская. 
Прадед Толстого был сподвижником Петра Первого, попал в опалу при Екатерине и был сослан в монастырь.
Дед - Илья Андреич Толстой. Жил в Москве, был членом английского клуба, хлебосольный, открытый, добродушный.
Отец исправил положение дел в семье, женившись на Марии Волконской.

\paragraph{}

Толстой учился в Казанском университете на отделении восточных языков, не сдал историю, перешел на юрфак, но
всё равно не окончил обучение. Уехал в Ясную поляну, начинает заниматься сельским хозяйством, стал помещиком.
Ничего не росло. С этого времени начинает вести дневник.

\paragraph{}

В 1850 году всё бросил и уехал на Кавказ к старшему брату Николаю. Пробыл там около полугода.
Впечатления от этой поездки лягут в основу повести ``Казаки''.
В 1852 году публикует своё первое произведение - повесть ``Детство''.

\paragraph{}

В 1854 году с Кавказа переводится в дунайские войска. Начинается крымская война. Толстой, будучи артиллеристом,
участвует в обороне Севастополя. Впоследствии об этих событиях он пишет рассказы.

\paragraph{}

В 1856 году Толстой выходит в отставку. Останавливается в Петербурге, знакомится с Тургеневым, но не ладит с ним
из-за разгульного образа жизни Толстого. Л.Н. решил вернуться в Ясную поляну с женой.
Его выбор пал на Екатерину, одну из дочерей Тютчева, его очень дальнего родственница. Та его отвергает.
Повторно посвататься он решает в 1860 году.

\paragraph{}

Для Толстого Петербург циничный, светский, темный, а Москва чистая, открытая, естественная. В 1862 году он
приезжает в Москву и знакомится с тремя дочерьми своего друга Берса. 23 сентября 1862 года он женился на средней
дочери Софьи Андреевной и сразу уехал в Ясную Поляну. Жили они в скромном домике. Старшая дочь Толстого -
Татьяна.

\paragraph{}

``Войну и мир'' Толстой писал с 1863 по 1869 гг. Предположительно, он писал её под впечатлением от романа
Виктора Гюго ``Отверженные''. Не случайно в романе так много вставок на французском. ``Война и мир'' повествует
о преддекабристской эпохе.

\paragraph{}

В 1872 году выходит роман ``Анна Каренина''. Сюжет схож с ``Грозой'' Островского: замужняя женщина
полюбила сильно и страстно. Автобиографический образ: Лёвин, который также как и Толстой мечтает найти молодую
жену и жить с ней в деревне.

\paragraph{}

В последующие годы в семье были несогласия, Толстой уехал из поместья, простудился и умер.

%HW: биографическая таблица Толстого, Война и мир 1 и 2 главы

\newpage

\begin{longtable}[c]{|p{3cm}|p{8cm}|}
  \caption{Биографическая таблица Л.Н. Толстого}\\
  \toprule
  \textbf{Дата} & \textbf{Событие}\\
  \hline
  \endfirsthead
  \toprule
  \textbf{Дата} & \textbf{Событие}\\
  \hline
  \endhead
  28 августа (9 сентября) 1828г. & Родился в имении Ясная Поляна Крапивинского
  уезда Тульской губернии в дворянской семье.\\
  \hline
  1837г. & Переезд семьи Толстых из Ясной Поляны в Москву. Смерть отца Толстого Николая Ильича.\\
  \hline
  1841г. & Смерть в Оптиной пустыни опекунши детей Толстых А. И. Остен-Сакен.
  Толстые переезжают из Москвы в Казань, к новой опекунше – П. И. Юшковой.\\
  \hline
  1844г. & Поступление в Казанский университет на восточный факультет, затем учеба на юридическом.
  Стремление постичь и понять мир – увлечение философией, изучение взглядов Руссо.\\
  \hline
  1847г. & Переезд в Ясную Поляну (без окончания университетского курса).
  Мучительные поиски смысла жизни. Проба пера – первые литературные наброски.\\
  \hline
  1849г. & Экзамены на степень кандидата в Петербургском университете.
  (Прекращены после удачной сдачи по двум предметам.)\\
  \hline
  1851г. & Написан рассказ «История вчерашнего дня». Начата повесть «Детство» (окончена в июле 1852 года).
  Отъезд на Кавказ на войну с горцами. Испытание самого себя. Война – осмысление пути формирования человека.\\
  \hline
  1852г. & Экзамен на звание юнкера, приказ о зачислении на военную службу фейерверкером 4-го класса.
  Написан рассказ «Набег». Завершена и напечатана (в № 9 «Современника») повесть "Детство" (начало трилогии).\\
  \hline
  1853г. & Начало работы над «Казаками» (завершена в 1862 году). Написан рассказ «Записки маркера».\\
  \hline
  1854г. & Повесть "Отрочество". Главный вопрос – каким надо быть? К чему стремиться?
  Процесс умственного и нравственного развития человека.
  Севастопольская эпопея. Перевод в Дунайскую армию, в сражающийся Севастополь
  после неудачного прошения об отставке.\\
  \hline
  1855г. & Написаны "Севастопольские рассказы" – гнев и боль о погибших, проклятие войне, жестокий реализм.\\
  \hline
  Ноябрь 1856г. & Увольнение из военной службы по личному прошению. "Утро помещика"
  (главное зло – жалкое, бедственное положение мужиков).\\
  \hline
  1857г. & Написана повесть "Юность" (завершение трилогии). Первое заграничное путешествие.\\
  \hline
  1859г. & Открытие школы в Ясной Поляне. Мысль о воспитании нового человека,
  создание "Азбуки" и книг для детей.\\
  \hline
  Сентябрь 1862г. & Женитьба на Софье Андреевне Берс; переезд в Ясную Поляну.\\
  \hline
  1863–1869гг. & Работа над романом-эпопеей "Война и мир".\\
  \hline
  1864–1865гг. & Выходит из печати первое Собрание сочинений Л. H. Толстого в двух томах.\\
  \hline
  1865–1866гг. & В ``Русском вестнике'' напечатаны две первые части будущей ``Войны и мира''
  под названием ``1805 год''.\\
  \hline
  1866г. & Знакомство с художником М. С. Башиловым, которому Толстой поручает иллюстрирование `` Войны и мира'' .\\
  \hline
  1867–1869гг. & Выход из печати двух отдельных изданий «Войны и мира».\\
  \hline
  1873–1877гг. & Работа над романом "Анна Каренина". Счастье личное и счастье народное.
  Жизнь семьи и жизнь России.\\
  \hline
  1875г. & Начало печатания ``Анны Карениной''  в журнале `` Русский вестник'' . 
  Во французском журнале ``Le temps'' напечатан перевод повести `` Два гусара''  с предисловием Тургенева,
  который писал, что по выходе `` Войны и мира''  Толстой `` решительно
  занимает первое место в расположении публики''.\\
  \hline
  1878г. & Отдельное издание романа ``Анна Каренина''.\\
  \hline
  1881г. & Переезд в Москву. Отречение от жизни дворянского круга. "Исповедь" (1879–1882).\\
  \hline
  1882г. & Участие в трехдневной московской переписи.
  Начата статья ``Так что же нам делать?'' (закончена в 1886 году).
  Покупка дома в Долго-Хамовническом переулке в Москве (ныне Дом-музей Л. Н. Толстого).
  Начата повесть ``Смерть Ивана Ильича'' (завершена в 1886 году).\\
  \hline
  1884г. & Портрет Толстого работы H. Н. Ге.
  Первая попытка уйти из Ясной Поляны. Основано издательство книг для народного чтения – ``Посредник''.\\
  \hline
  1886г. & Знакомство с В. Г. Короленко.
  Написана драма для народного театра – ``Власть тьмы'' (запрещена к постановке).
  Начата комедия ``Плоды просвещения'' (закончена в 1890 году).\\
  \hline
  1887г. & Знакомство с Н. С. Лесковым.
  Начата ``Крейцерова соната'' (закончена в 1889 году).\\
  \hline
  1889–1899гг. & Роман "Воскресение". Протест против беззакония и лжи общества.\\
  \hline
  1891–1893гг. & Организация помощи голодающим крестьянам Рязанской губернии. Статьи о голоде.\\
  \hline
  1895г. & Знакомство с А. П. Чеховым. Представление ``Власти тьмы'' в Малом театре.
  Написана статья ``Стыдно'' – протест против телесных наказаний крестьян.\\
  \hline
  1896г. & Начата повесть ``Хаджи Мурат'' (работа продолжалась до 1904 года).\\
  \hline
  1898г. & Организация помощи голодающим крестьянам Тульской губернии. Статья ``Голод или не голод?''.
  Решение напечатать ``Отца Сергия'' и ``Воскресение'' в пользу духоборов, переселяющихся в Канаду.
  В Ясной Поляне Л. О. Пастернак, иллюстрирующий ``Воскресение''.\\
  \hline
  1899г. & В журнале ``Нива'' печатается роман ``Воскресение''.\\
  \hline
  24 февраля 1901г. & Официальное отлучение от церкви. В связи с болезнью отъезд в Крым, в Гаспру.\\
  \hline
  1902г. & Возвращение в Ясную Поляну.\\
  \hline
  1903г. & Рассказ "После бала".\\
  \hline
  1910г. (ночь с 27 на 28 октября)  & Уход из Ясной Поляны.\\
  \hline
  7(20) ноября & Умер на станции Астапово, похоронен в Ясной Поляне.\\
  \bottomrule
\end{longtable}

\newpage

\noindent\makebox[\linewidth]{\rule{\paperwidth}{0.4pt}}
\section{(11.09.2018)}
\noindent\makebox[\linewidth]{\rule{\paperwidth}{0.4pt}}
\subsection{История создания романа ``Война и мир''}

В 1856 году Александр II объявил амнистию многих декабристов.
Толстой увлёкся этой темой и решил написать роман о восстании декабристов 1825 года. Исстоки движения декабристов
относятся к войне 1812 года, когда народ надеялся, что власть отменит крепостное право. 

\paragraph{}

Начало действия романа ``Война и мир'' происходит в 1805-1807 годы, когда Россия проиграла в войне с Наполеоном.
Толстой считал, что ``Нельзя говорить о победах, не показав поражений.''

\paragraph{}
\noindent\makebox[\linewidth][c]{\textbf{Смысл названия}}
\paragraph{}
\noindent\makebox[\linewidth][c]{
  \begin{minipage}{0.5\linewidth}
    \noindent\makebox[\linewidth][c]{\textbf{Война}}
  \end{minipage}
  \begin{minipage}{0.5\linewidth}
    \noindent\makebox[\linewidth][c]{\textbf{Мир}}
  \end{minipage}
}
\paragraph{}
\noindent\makebox[\linewidth][c]{
\begin{minipage}{0.5\linewidth}
  \begin{enumerate}
  \item
    Вооруженное противостояние государтсв.
  \item
    Вражда между людьми.
  \item
    Внутренний конфликт.
  \end{enumerate}
\end{minipage}

\begin{minipage}{0.5\linewidth}
  \begin{enumerate}
  \item
    Спокойная жизнь людей и государства.
  \item
    Светское общество.
  \item
    Мир в душе, гармония.
  \item
    Убеждения человека.
  \item
    Религизное значение.
  \item
    Общество, народ.
  \end{enumerate}
\end{minipage}}
\paragraph{}

Основной художественный приём в романе --- антитеза (противопоставление).
Противопоставление дается в:
\begin{itemize}
\item
  Названии (``Война и мир'')
\item
  В военных компаниях (1805-1807 гг. и 1812 г.)
\item
  В полководцах (Наполеон и Кутузов)
\item
  В городах (Петербург и Москва)
\item
  В семьях (Курагины и Болконские, Ростовы)
\item
  В образах (Пьер Безухов и князь Андрей
\item
  В композиции
\end{itemize}
\paragraph{}
Жанр: роман-эпопея. Множество сюжетных линий, огромное количество персонажей, охват событий
с 1805 по 1821 год,

\paragraph{Роман-эпопея:}

\begin{enumerate}
\item
  Исторический
\item
  Философский (проблемы веры и неверия, жизни и смерти, смысла жизни, добра и зла)
\item
  Психологический
\end{enumerate}

\subsection{Война и мир как психологический роман}

Чернышевский: ``Психологизм Толстого - это диалектика души''.
\paragraph{}

Приёмы:

\begin{itemize}
\item
  Психологический портрет
\item
  Внутренний монолог
\item
  Авторский комментарий
\item
  Использование пейзажа
\item
  Использование иностранного языка
\item
  Письма и сны
\end{itemize}

\newpage
\noindent\makebox[\linewidth]{\rule{\paperwidth}{0.4pt}}
\section{(14.09.18)}
\noindent\makebox[\linewidth]{\rule{\paperwidth}{0.4pt}}
\paragraph{}

\paragraph{Направления сочинения (сдать до конца сентября):}

\begin{enumerate}
\item
  Почему проблема отцов и детей является вечной?
\item
  Помечу часто мечта разбивается о реальность?
\item
  Может ли быть месть праведной?
\item
  В чём состоит загадка подлинного таланта?
\item
  Согласны ли вы с утверждением Достоевского о том, что в каждом человеке есть идеал содомский и идеал мадонны?
\end{enumerate}
\paragraph{}

\subsection{Пути исканий героев романа}

\paragraph{Князь Андрей}
27 лет, ведет светский образ жизни, который ему надоел. Он мечтает о славе Наполеона, о своём тулоне.
Идет на войну адъютантой Кутузова. Женат, скоро станет отцом, но он разочарован в браке. Невысокого роста
красивый молодой человек с сухими чертами лица. Князь Андрей мечтает вернуться домой и жить для семьи, но
этим мечтам не суждено сбыться: жена князя умирает во время родов, и он испытывает огромное чувство вины перед ней.
Два года он живет в Лысых горах, занимается благоустройством имения сына. Его жизненное кредо теперь:
"Нужно жить для себя, избегая двух зол: угрызений совести и болезней". В этот период происходит встреча князя
Андрея с Пьером. Между друзьями происходит спор о вере, о добре и зле, о смысле жизни. И хотя Андрей Болконский
скептически относится к высказываниям друга, эта встреча пробуждает в его душе нравственное беспокойство.
В конце встречи князь преображается внешне: его потухший мертвый взгляд становится лучистым, детским, нежным.
И в первый раз за всё это время он вновь видит высокое небо. После встречи с Пьером для князя Андрея началась
новая эпоха его нравственных поисков. После встречи с Наташей Ростовой в Отрадном в душе князя наметился перелом.
Он чувствует себя помолодевшим, возродившимся к новой жизни. Он принимает решение ехать в Петербург и поступает
на государственную службу в комиссию Сперанского. Но вскоре Болконский разочаровывается в этой деятельности, т.к.
законы остаются лишь проектами на бумаге. На новогоднем балу князь Адрей увлекается Наташей Ростовой.
Он вновь мечтает обрести семейное счастье, и делает предложение Наташе. По настоянию отца свадьба отложена на год.
Князь Андрей уезжает на лечение заграницу. Но Наташа, нетерпеливая, порывистая, жаждущая любви, не выдержала этого
испытания. Она пытается бежать с Анатолем Курагиным, и отказывает князю Андрею. Разрыв с Наташей стал для князя
причиной самого тяжелого духовного кризиса. Злоба и неотмщённое самолюбие отравляли его жизнь. Он вновь идёт на
военную службу, но им движут личные мстительные интересы --- найти обидчика и вызвать его на дуэль. 
\paragraph{Пьер Безухов}
20 лет, получил образование заграницей, приехал в Петербург выбирать карьеру. Незаконнорожденный сын графа Безухова.
Ведет распутный, рассеянный образ жизни в компании Анатолия Курагина и Долохова. После смерти отца становится
богатым женихом. Близорукий, толстый, массивный, с короткой стрижкой. Женился на Элен. Развелся.
На станции в Торшке Пьер встречается с членом масонской ложи Иосифом Боздеевым, после чего вступает в ложу вольных
каменщиков. Масонство даёт ему нравственную опору. Его привлекают идеи братства, добра, самосовершенствования.
Пьер упорно работает над собой (ведёт дневник, борется с пороками). Слова "надо жить", "надо любить", "надо верить"
становятся жизненным кредом Пьера. Он мирится с женой и занимается масонской деятельностью. Вскоре Пьер понимает,
что многие знатные люди стремятся попасть в ложу, чтобы завязать нужные связи, получить протекцию. Это заставляет
Пьера разочароваться в масонской деятельности. Он вновь погружается в праздную жизнь. И только любовь к Наташе
помогает ему преодолеть этот кризис.
\paragraph{Николай Ростов}
20 лет, граф. Собирался поступать в московский университет, но принял решение идти в армию в гусарский павлоградский
полк. Влюблён в Соню, племянницу Ростовых. Невысокого роста, курчавый молодой человек с усиками.
В 1806 году Николай уезжает в Москву в отпуск.
Теперь это бравый офицер, участвовавший в нескольких сражениях и преодолевший свою неуверенность. Его принимают
во всех домах, он радуется жизни. После ссоры Долохова и Безухова становится секундантом Долохова.
Происходит сближение с Долоховым, которое заканчивается проигрышем 43000 рублей (из-за Сони). Ростов едет в полк,
намереваясь выплачивать почти всё жалование родителям в счёт долга. Во время отпуска Николай делает предложение
Соне. Но мать недовольна этим, так как Соня бесприданница. Происходит ссора с матерью, и Николай уезжает в полк
в расстроенных чувствах.

\newpage
\noindent\makebox[\linewidth]{\rule{\paperwidth}{0.4pt}}
\section{(16.10.18)}
\noindent\makebox[\linewidth]{\rule{\paperwidth}{0.4pt}}

\subsection{Темы сочинения}

\begin{enumerate}
\item
  Согласны ли вы с высказыванием Пушкина "Неуважение к предкам есть первый признак безнравственности".
\item
  Мечтательность: это положительное или отрицательное качество человека?
\item
  Как вы понимаете выражение "сладкая месть"?
\item
  Может ли ремесленник стать художником?
\item
  Согласны ли вы с мнением Гейне, что доброта лучше красоты?
\end{enumerate}



\end{document}

