\documentclass{article}
\author{SDL}

\usepackage{subcaption}
\usepackage{titlesec}
\usepackage{longtable}
\usepackage{booktabs}
\usepackage[T1,T2A]{fontenc}
\usepackage[utf8]{inputenc}
\usepackage[english,russian]{babel}

\titleformat{\section}
            {\normalfont\Large\bfseries}
            {}
            {0pt}
            {Урок \thesection\quad}

\begin{document}

\noindent\makebox[\linewidth]{\rule{\paperwidth}{0.4pt}}
\section{(04.09.2018)}
\noindent\makebox[\linewidth]{\rule{\paperwidth}{0.4pt}}


\subsection{Темы итогового сочинения}
\subsubsection{Отцы и дети}
Отцы и дети, Война и мир, Капитанская дочка, Горе от ума, Недоросль, Гроза, Обломов
\subsubsection{Мечта и реальность}
Белые ночи, Преступление и наказание, Алые паруса
\subsubsection{Месть и великодушие}
Война и мир - взаимоотношения Пьера и Долохова, Капитанская дочка, Выстрел (Пушкин), Мастер и Маргарита
\subsubsection{Искусство и ремесло}
Левша, Тупейный художник, Очарованный странник, Мастер и Маргарита, Портрет (Гоголь),
Разговор книгопродавца с поэтом, Лирическое отступление из Мертвых душ,
Блажен незлобливый поэт, Творчество (Ахматова)
\subsubsection{Доброта и жестокость}
\paragraph{Во взаимоотношениях между людьми}
Капитанская дочка, Преступление и наказание, Война и мир, Чучело
\paragraph{В отношениях с животными}
Зачем я убил коростеля, Гуси в полынье, Конь с розовой гривой

\subsection{Критерии оценивания итогового сочинения}
\begin{enumerate}
\item
  Соответствие темы
\item
  Наличие литературных аргументов
\item
  Наличие речевых ошибок
\item
  Наличие грамматических ошибок
\item
  Орфография и пунктуация
\end{enumerate}
Объем сочинения - 350 слов.
Зачет идёт при соотсветствии хотя бы трём критериям.
\subsection{Лев Николаевич Толстой (1828-1910)}

По происхождению аристократ. Со стороны матери родственники были Волконские и Трубецкие.
По линии отца Горчаковы и Толстые. Дед писателя Волконский был военным, очень знатным человеком, своенравным.
Женился на Надежде Сергеевне Трубецкой, у них родилась дочь Мария Николаевна Волконская, мать Толстого.

\paragraph{}

В семье было пятеро детей, Л.Н. был четвертым. Мать умерла при родах пятого ребенка, когда Толстому было два года.
За воспитание детей отвечала дальняя родственница Татьяна Ергольская. 
Прадед Толстого был сподвижником Петра Первого, попал в опалу при Екатерине и был сослан в монастырь.
Дед - Илья Андреич Толстой. Жил в Москве, был членом английского клуба, хлебосольный, открытый, добродушный.
Отец исправил положение дел в семье, женившись на Марии Волконской.

\paragraph{}

Толстой учился в Казанском университете на отделении восточных языков, не сдал историю, перешел на юрфак, но
всё равно не окончил обучение. Уехал в Ясную поляну, начинает заниматься сельским хозяйством, стал помещиком.
Ничего не росло. С этого времени начинает вести дневник.

\paragraph{}

В 1850 году всё бросил и уехал на Кавказ к старшему брату Николаю. Пробыл там около полугода.
Впечатления от этой поездки лягут в основу повести ``Казаки''.
В 1852 году публикует своё первое произведение - повесть ``Детство''.

\paragraph{}

В 1854 году с Кавказа переводится в дунайские войска. Начинается крымская война. Толстой, будучи артиллеристом,
участвует в обороне Севастополя. Впоследствии об этих событиях он пишет рассказы.

\paragraph{}

В 1856 году Толстой выходит в отставку. Останавливается в Петербурге, знакомится с Тургеневым, но не ладит с ним
из-за разгульного образа жизни Толстого. Л.Н. решил вернуться в Ясную поляну с женой.
Его выбор пал на Екатерину, одну из дочерей Тютчева, его очень дальнего родственница. Та его отвергает.

%HW: биографическая таблица Толстого, Война и мир 1 и 2 главы

\newpage

\begin{longtable}[c]{|p{3cm}|p{8cm}|}
  \caption{Биографическая таблица Л.Н. Толстого}\\
  \endfirsthead
  \toprule
  \textbf{Дата} & \textbf{Событие}\\
  \midrule
  \endhead
  28 августа (9 сентября) 1828г. & Родился в имении Ясная Поляна Крапивинского
  уезда Тульской губернии в дворянской семье.\\
  \hline
  1837г. & Переезд семьи Толстых из Ясной Поляны в Москву. Смерть отца Толстого Николая Ильича.\\
  \hline
  1841г. & Смерть в Оптиной пустыни опекунши детей Толстых А. И. Остен-Сакен.
  Толстые переезжают из Москвы в Казань, к новой опекунше – П. И. Юшковой.\\
  \hline
  1844г. & Поступление в Казанский университет на восточный факультет, затем учеба на юридическом.
  Стремление постичь и понять мир – увлечение философией, изучение взглядов Руссо.\\
  \hline
  1847г. & Переезд в Ясную Поляну (без окончания университетского курса).
  Мучительные поиски смысла жизни. Проба пера – первые литературные наброски.\\
  \hline
  1849г. & Экзамены на степень кандидата в Петербургском университете.
  (Прекращены после удачной сдачи по двум предметам.)\\
  \hline
  1851г. & Написан рассказ «История вчерашнего дня». Начата повесть «Детство» (окончена в июле 1852 года).
  Отъезд на Кавказ на войну с горцами. Испытание самого себя. Война – осмысление пути формирования человека.\\
  \hline
  1852г. & Экзамен на звание юнкера, приказ о зачислении на военную службу фейерверкером 4-го класса.
  Написан рассказ «Набег». Завершена и напечатана (в № 9 «Современника») повесть "Детство" (начало трилогии).\\
  \hline
  1853г. & Начало работы над «Казаками» (завершена в 1862 году). Написан рассказ «Записки маркера».\\
  \hline
  1854г. & Повесть "Отрочество". Главный вопрос – каким надо быть? К чему стремиться?
  Процесс умственного и нравственного развития человека.
  Севастопольская эпопея. Перевод в Дунайскую армию, в сражающийся Севастополь
  после неудачного прошения об отставке.\\
  \hline
  1855г. & Написаны "Севастопольские рассказы" – гнев и боль о погибших, проклятие войне, жестокий реализм.\\
  \hline
  Ноябрь 1856г. & Увольнение из военной службы по личному прошению. "Утро помещика"
  (главное зло – жалкое, бедственное положение мужиков).\\
  \hline
  1857г. & Написана повесть "Юность" (завершение трилогии). Первое заграничное путешествие.\\
  \hline
  1859г. & Открытие школы в Ясной Поляне. Мысль о воспитании нового человека,
  создание "Азбуки" и книг для детей.\\
  \hline
  Сентябрь 1862г. & Женитьба на Софье Андреевне Берс; переезд в Ясную Поляну.\\
  \hline
  1863–1869гг. & Работа над романом-эпопеей "Война и мир".\\
  \hline
  1864–1865гг. & Выходит из печати первое Собрание сочинений Л. H. Толстого в двух томах.\\
  \hline
  1865–1866гг. & В ``Русском вестнике'' напечатаны две первые части будущей ``Войны и мира''
  под названием ``1805 год''.\\
  \hline
  1866г. & Знакомство с художником М. С. Башиловым, которому Толстой поручает иллюстрирование `` Войны и мира'' .\\
  \hline
  1867–1869гг. & Выход из печати двух отдельных изданий «Войны и мира».\\
  \hline
  1873–1877гг. & Работа над романом "Анна Каренина". Счастье личное и счастье народное.
  Жизнь семьи и жизнь России.\\
  \hline
  1875г. & Начало печатания ``Анны Карениной''  в журнале `` Русский вестник'' . 
  Во французском журнале ``Le temps'' напечатан перевод повести `` Два гусара''  с предисловием Тургенева,
  который писал, что по выходе `` Войны и мира''  Толстой `` решительно
  занимает первое место в расположении публики''.\\
  \hline
  1878г. & Отдельное издание романа ``Анна Каренина''.\\
  \hline
  1881г. & Переезд в Москву. Отречение от жизни дворянского круга. "Исповедь" (1879–1882).\\
  \hline
  1882г. & Участие в трехдневной московской переписи.
  Начата статья ``Так что же нам делать?'' (закончена в 1886 году).
  Покупка дома в Долго-Хамовническом переулке в Москве (ныне Дом-музей Л. Н. Толстого).
  Начата повесть ``Смерть Ивана Ильича'' (завершена в 1886 году).\\
  \hline
  1884г. & Портрет Толстого работы H. Н. Ге.
  Первая попытка уйти из Ясной Поляны. Основано издательство книг для народного чтения – ``Посредник''.\\
  \hline
  1886г. & Знакомство с В. Г. Короленко.
  Написана драма для народного театра – ``Власть тьмы'' (запрещена к постановке).
  Начата комедия ``Плоды просвещения'' (закончена в 1890 году).\\
  \hline
  1887г. & Знакомство с Н. С. Лесковым.
  Начата ``Крейцерова соната'' (закончена в 1889 году).\\
  \hline
  1889–1899гг. & Роман "Воскресение". Протест против беззакония и лжи общества.\\
  \hline
  1891–1893гг. & Организация помощи голодающим крестьянам Рязанской губернии. Статьи о голоде.\\
  \hline
  1895г. & Знакомство с А. П. Чеховым. Представление ``Власти тьмы'' в Малом театре.
  Написана статья ``Стыдно'' – протест против телесных наказаний крестьян.\\
  \hline
  1896г. & Начата повесть ``Хаджи Мурат'' (работа продолжалась до 1904 года).\\
  \hline
  1898г. & Организация помощи голодающим крестьянам Тульской губернии. Статья ``Голод или не голод?''.
  Решение напечатать ``Отца Сергия'' и ``Воскресение'' в пользу духоборов, переселяющихся в Канаду.
  В Ясной Поляне Л. О. Пастернак, иллюстрирующий ``Воскресение''.\\
  \hline
  1899г. & В журнале ``Нива'' печатается роман ``Воскресение''.\\
  \hline
  24 февраля 1901г. & Официальное отлучение от церкви. В связи с болезнью отъезд в Крым, в Гаспру.\\
  \hline
  1902г. & Возвращение в Ясную Поляну.\\
  \hline
  1903г. & Рассказ "После бала".\\
  \hline
  1910г. (ночь с 27 на 28 октября)  & Уход из Ясной Поляны.\\
  \hline
  7(20) ноября & Умер на станции Астапово, похоронен в Ясной Поляне.\\
  \bottomrule
\end{longtable}

\end{document}

